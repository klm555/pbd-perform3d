%% Generated by Sphinx.
\def\sphinxdocclass{report}
\documentclass[a4paper,10pt,korean]{sphinxmanual}
\ifdefined\pdfpxdimen
   \let\sphinxpxdimen\pdfpxdimen\else\newdimen\sphinxpxdimen
\fi \sphinxpxdimen=.75bp\relax
\ifdefined\pdfimageresolution
    \pdfimageresolution= \numexpr \dimexpr1in\relax/\sphinxpxdimen\relax
\fi
%% let collapsible pdf bookmarks panel have high depth per default
\PassOptionsToPackage{bookmarksdepth=5}{hyperref}
%% turn off hyperref patch of \index as sphinx.xdy xindy module takes care of
%% suitable \hyperpage mark-up, working around hyperref-xindy incompatibility
\PassOptionsToPackage{hyperindex=false}{hyperref}
%% memoir class requires extra handling
\makeatletter\@ifclassloaded{memoir}
{\ifdefined\memhyperindexfalse\memhyperindexfalse\fi}{}\makeatother

\PassOptionsToPackage{warn}{textcomp}

\catcode`^^^^00a0\active\protected\def^^^^00a0{\leavevmode\nobreak\ }
\usepackage{cmap}
\usepackage{fontspec}
\defaultfontfeatures[\rmfamily,\sffamily,\ttfamily]{}
\usepackage{amsmath,amssymb,amstext}
\usepackage{polyglossia}
\setmainlanguage{korean}



\usepackage{kotex}
\usepackage{setspace}
\singlespacing
\usepackage[skip=10pt plus1pt, indent=20pt]{parskip}

\setmainfont{Noto Serif KR}
\setsansfont{Noto Sans KR}
\setmonofont{Noto Sans KR}



\usepackage[Sonny]{fncychap}
\ChNameVar{\Large\normalfont\sffamily}
\ChTitleVar{\Large\normalfont\sffamily}
\usepackage{sphinx}

\fvset{fontsize=\small}
\usepackage{geometry}

\usepackage{fontawesome}

% Include hyperref last.
\usepackage{hyperref}
% Fix anchor placement for figures with captions.
\usepackage{hypcap}% it must be loaded after hyperref.
% Set up styles of URL: it should be placed after hyperref.
\urlstyle{same}

\addto\captionskorean{\renewcommand{\contentsname}{STEPS}}

\usepackage{sphinxmessages}
\setcounter{tocdepth}{0}



\title{성능기반 내진설계 업무절차서}
\date{2023년 02월 01일}
\release{v1.0}
\author{CNP Dongyang}
\newcommand{\sphinxlogo}{\vbox{}}
\renewcommand{\releasename}{릴리스}
\makeindex
\begin{document}

\pagestyle{empty}
\sphinxmaketitle
\pagestyle{plain}
\sphinxtableofcontents
\pagestyle{normal}
\phantomsection\label{\detokenize{index::doc}}


\sphinxAtStartPar
본 사이트는 성능기반 내진설계를 위한 업무절차서, 테크니컬 노트, PBD\_p3d 프로그램 매뉴얼(for Developers) 등을 제공합니다.

\sphinxstepscope


\chapter{성능기반 내진설계 업무절차서 v1.0}
\label{\detokenize{0_pbd_p3d_manual:v1-0}}\label{\detokenize{0_pbd_p3d_manual::doc}}
\sphinxAtStartPar
빠르게 배우고 실무에 뛰어들 수 있도록

\sphinxAtStartPar
본 업무절차서는 성능기반 내진설계를 \sphinxstylestrong{정확하고, 빠르고, 편리하게} 진행하기 위해 작성되었습니다.

\sphinxstepscope


\section{탄성설계 모델, 레퍼런스 모델 생성}
\label{\detokenize{1_ref_model:id1}}\label{\detokenize{1_ref_model::doc}}
\sphinxAtStartPar
\DUrole{sd-sphinx-override,sd-badge,sd-bg-danger,sd-bg-text-danger}{Midas GEN}

\sphinxAtStartPar
해외 프로젝트를 제외하면 대부분의 프로젝트에서는 탄성설계를 Midas Gen으로 진행합니다. 반면, 저희가 성능설계 모델링에 사용할 프로그램은 Perform\sphinxhyphen{}3D입니다.
따라서 Midas Gen에서 Perform\sphinxhyphen{}3D로 모델링 정보를 변환하는 과정이 필요한데, 두 프로그램은 엄연히 다른 프로그램이기 떄문에 모든 정보를 변환할 수 없습니다.
다만 탄성설계 모델을 최대한 다듬고 수정하여 변환한다면, 최대한 많은 정보들을 변환할 수 있습니다.

\begin{figure}[htbp]
\centering
\capstart

\noindent\sphinxincludegraphics{{유효강성}.png}
\caption{\sphinxstyleemphasis{Fig1. 탄성설계 모델 \sphinxhyphen{} 레퍼런스 모델 \sphinxhyphen{} 성능설계 모델}}\label{\detokenize{1_ref_model:id5}}\end{figure}


\subsection{탄성설계 모델}
\label{\detokenize{1_ref_model:id2}}
\sphinxAtStartPar
탄성설계 모델은 \sphinxstylestrong{탄성설계 과정에서 사용된 최종 Midas Gen 모델} 입니다.
대부분의 탄성설계 모델은 이미 완성된 상태로 전달받기 때문에 따로 생성할 필요가 없으나,
이 후 만들어질 레퍼런스 모델, 성능설계 모델과의 비교 검증을 위해 몇 가지 사항을 설정해 주어야합니다.


\subsection{레퍼런스 모델}
\label{\detokenize{1_ref_model:id3}}
\sphinxAtStartPar
앞서 설명한 바와 같이, Perform\sphinxhyphen{}3D로 최대한 많은 정보들을 변환하기 위해 탄성설계 모델을 수정하는 과정이 필요합니다.
저희는 \sphinxstylestrong{Perform\sphinxhyphen{}3D로 변환하기 전, 최종 Midas Gen 모델} 을 레퍼런스 모델 이라고 이름지었습니다.
탄성설계 모델에서 성능설계 모델로 변환할 때 만들어지는 중간 모델이라고 생각하시면 됩니다.


\subsection{성능설계 모델}
\label{\detokenize{1_ref_model:id4}}
\sphinxAtStartPar
성능설계 모델은 성능기반 내진설계에 필요한 Perform\sphinxhyphen{}3D 모델 입니다.
이번 장에서 만들어지는 레퍼런스 모델을 기반으로 생성합니다.

\begin{sphinxadmonition}{note}{참고:}
\sphinxAtStartPar
레퍼런스 모델은 아래와 같은 이유때문에 중요합니다.
\begin{enumerate}
\sphinxsetlistlabels{\arabic}{enumi}{enumii}{}{.}%
\item {} 
\sphinxAtStartPar
Perform\sphinxhyphen{}3D로 Import하기 위한 최종 모델이기때문에, 이 모델이 정확하지 않으면 성능설계 모델도 부정확하게 모델링됩니다.

\item {} 
\sphinxAtStartPar
탄성설계 모델에서 성능설계 모델로 변환하는 과정에서 변경 사항이 많다면, 레퍼런스 모델이 비교 검증에 가장 중요한 역할을 하는 모델이 됩니다.(?)

\end{enumerate}
\end{sphinxadmonition}

\sphinxAtStartPar
탄성설계 모델과 레퍼런스 모델의 생성 과정은 아래의 단계를 거칩니다.

\sphinxstepscope


\subsubsection{단위 설정}
\label{\detokenize{1_unit_setting:id1}}\label{\detokenize{1_unit_setting::doc}}
\sphinxAtStartPar
모든 모델링에서 가장 먼저 해야할 것은 단위를 설정하는 것입니다. 성능기반 내진설계에서도 마찬가지입니다.
특히 본 매뉴얼에서 소개할 성능기반 내진설계는 여러가지 소프트웨어를 활용하기 때문에, 각각의 소프트웨어마다 단위를 알맞게 변환, 통일하는 것이
매우 중요합니다. 저희는 단위 변환으로 인한 사용자들의 실수와 혼란을 방지하고, 단위 변환에 필요한 프로세스를 줄이기 위해 단위를 통일하여 사용할 것입니다.
저희가 앞으로 사용할 단위는 \sphinxstylestrong{kN, mm} 입니다.

\begin{sphinxuseclass}{sd-card}
\begin{sphinxuseclass}{sd-sphinx-override}
\begin{sphinxuseclass}{sd-mb-3}
\begin{sphinxuseclass}{sd-shadow-sm}
\begin{sphinxuseclass}{sd-card-body}
\sphinxAtStartPar
새로 생성한 탄성설계 모델의 단위를 kN, mm로 설정합니다.

\end{sphinxuseclass}
\end{sphinxuseclass}
\end{sphinxuseclass}
\end{sphinxuseclass}
\end{sphinxuseclass}
\sphinxstepscope


\subsubsection{유효강성 설정}
\label{\detokenize{1_stiffness_setting:id1}}\label{\detokenize{1_stiffness_setting::doc}}
\sphinxAtStartPar
성능기반 내진설계에서의 유효강성은 아래의 표%
\begin{footnote}[1]\sphinxAtStartFootnote
대한건축학회, 철근콘크리트 건축구조물의 성능기반 내진설계 지침(2021) {[}표 6\sphinxhyphen{}1{]}
%
\end{footnote} 에 따릅니다.

\begin{figure}[htbp]
\centering
\capstart

\noindent\sphinxincludegraphics{{유효강성}.png}
\caption{\sphinxstyleemphasis{Fig2. 비선형모델의 유효강성}}\label{\detokenize{1_stiffness_setting:id8}}\end{figure}

\begin{sphinxadmonition}{note}{참고:}
\sphinxAtStartPar
유효강성은 Midas Gen에서 Perform\sphinxhyphen{}3D로 Import되는 항목이 아니지만,
이 후 생성할 레퍼런스 모델, 성능설계 모델과의 비교 검증을 위해 사용됩니다.
\end{sphinxadmonition}

\sphinxAtStartPar
위의 표를 참고하여 Midas Gen에서 유효강성을 변경합니다. 다만 전단강성 \(GA_W\) 의 경우, 계산에 필요한 단면적 \(A_W\) 가
유효단면적( \(A_e\) )이 아닌 전체단면적( \(A_g\) )임에 주의해야 합니다.
Midas Gen에서는 유효단면적(\(A_e = \frac{5}{6}A_g\) ; 모든 보의 단면적은 직사각형으로 가정함)을 자동으로 계산하여 사용하므로,
역수인 \(\frac{6}{5}(\approx 1.2)\) 를 곱하여 전체단면적을 만들어 사용합니다.


\paragraph{연결보}
\label{\detokenize{1_stiffness_setting:id3}}
\sphinxAtStartPar
연결보의 유효강성은 아래의 절차에 따라 변경, 추가합니다.

\begin{sphinxuseclass}{sd-card}
\begin{sphinxuseclass}{sd-sphinx-override}
\begin{sphinxuseclass}{sd-mb-3}
\begin{sphinxuseclass}{sd-shadow-sm}
\begin{sphinxuseclass}{sd-card-body}\begin{enumerate}
\sphinxsetlistlabels{\arabic}{enumi}{enumii}{}{.}%
\item {} 
\sphinxAtStartPar
Midas Gen에서 \sphinxguilabel{Properties} \sphinxhyphen{} \sphinxguilabel{Scale Factor} \sphinxhyphen{} \sphinxguilabel{Section Stiffness Scale Factor} 를 클릭합니다.

\end{enumerate}

\end{sphinxuseclass}
\end{sphinxuseclass}
\end{sphinxuseclass}
\end{sphinxuseclass}
\end{sphinxuseclass}
\begin{sphinxuseclass}{sd-card}
\begin{sphinxuseclass}{sd-sphinx-override}
\begin{sphinxuseclass}{sd-mb-3}
\begin{sphinxuseclass}{sd-shadow-sm}
\begin{sphinxuseclass}{sd-card-body}\begin{enumerate}
\sphinxsetlistlabels{\arabic}{enumi}{enumii}{}{.}%
\setcounter{enumi}{1}
\item {} 
\sphinxAtStartPar
\sphinxguilabel{Section Stiffness Scale Factor} 창에서 변경, 추가할 연결보의 Section을 선택한 후, Scale Factor를 변경하여 줍니다.
휨강성은 \(0.3EI\) 이므로, \(I_{yy}, I_{zz}\) 에 각각 \(0.3\) 을 입력합니다.

\end{enumerate}

\begin{figure}[htbp]
\centering

\noindent\sphinxincludegraphics[scale=0.6]{{c_beam_bending_유효강성}.png}
\end{figure}

\sphinxAtStartPar
입력 후, \sphinxkeyboard{\sphinxupquote{Add/Replace}} 버튼을 누릅니다.

\end{sphinxuseclass}
\end{sphinxuseclass}
\end{sphinxuseclass}
\end{sphinxuseclass}
\end{sphinxuseclass}
\begin{sphinxuseclass}{sd-card}
\begin{sphinxuseclass}{sd-sphinx-override}
\begin{sphinxuseclass}{sd-mb-3}
\begin{sphinxuseclass}{sd-shadow-sm}
\begin{sphinxuseclass}{sd-card-body}\begin{enumerate}
\sphinxsetlistlabels{\arabic}{enumi}{enumii}{}{.}%
\setcounter{enumi}{2}
\item {} 
\sphinxAtStartPar
전단강성은 \(0.04(\frac{l}{h})GA\) 이므로, \(A_{sy}, I_{sz}\) 의 값을 변경해야 합니다.
연결보의 길이( \(l\) )와 깊이( \(h\) )를 확인한 후, \(0.04(\frac{l}{h})\) 를 계산합니다.
위의 설명과 같이, \(1.2\) 를 곱합니다.

\end{enumerate}

\begin{figure}[htbp]
\centering

\noindent\sphinxincludegraphics[scale=0.6]{{c_beam_shear_유효강성}.png}
\end{figure}

\end{sphinxuseclass}
\end{sphinxuseclass}
\end{sphinxuseclass}
\end{sphinxuseclass}
\end{sphinxuseclass}
\begin{sphinxadmonition}{warning}{경고:}
\sphinxAtStartPar
Midas Gen 모델링 과정에서 짧은 벽을 생략하는 경우, 연결보의 길이가 길게 모델링되는 경우가 있습니다.
따라서 \sphinxstylestrong{도면을 확인} 하여 정확한 연결보의 길이를 이용해 계산합니다.
\end{sphinxadmonition}

\begin{sphinxuseclass}{sd-card}
\begin{sphinxuseclass}{sd-sphinx-override}
\begin{sphinxuseclass}{sd-mb-3}
\begin{sphinxuseclass}{sd-shadow-sm}
\begin{sphinxuseclass}{sd-card-body}\begin{enumerate}
\sphinxsetlistlabels{\arabic}{enumi}{enumii}{}{.}%
\setcounter{enumi}{3}
\item {} 
\sphinxAtStartPar
모든 연결보의 유효강성을 변경, 추가한 후, \sphinxkeyboard{\sphinxupquote{Close}} 버튼을 누릅니다.

\end{enumerate}

\end{sphinxuseclass}
\end{sphinxuseclass}
\end{sphinxuseclass}
\end{sphinxuseclass}
\end{sphinxuseclass}

\paragraph{보, 기둥}
\label{\detokenize{1_stiffness_setting:id4}}
\sphinxAtStartPar
연결보와 동일한 방식으로 \sphinxguilabel{Section Stiffness Scale Factor} 에서 유효강성을 변경, 추가합니다.
기둥의 휨강성 역시, Data Conversion Sheet에서 자동으로 계산된 값을 사용할 수 있습니다.


\paragraph{벽체}
\label{\detokenize{1_stiffness_setting:id5}}
\sphinxstepscope


\subsubsection{질량 및 우발편심 설정}
\label{\detokenize{1_mass_ecc_setting:id1}}\label{\detokenize{1_mass_ecc_setting::doc}}

\paragraph{질량 설정}
\label{\detokenize{1_mass_ecc_setting:id2}}
\sphinxAtStartPar
비선형 해석을 위해 입력되는 질량은 고정하중에 해당하는 질량만 고려됩니다.%
\begin{footnote}[1]\sphinxAtStartFootnote
대한건축학회, 철근콘크리트 건축구조물의 성능기반 내진설계 지침(2021), 4.1\sphinxhyphen{}(4)
%
\end{footnote}
Midas Gen에서 이를 미리 설정하여

\sphinxAtStartPar
그러나

\begin{sphinxuseclass}{sd-card}
\begin{sphinxuseclass}{sd-sphinx-override}
\begin{sphinxuseclass}{sd-mb-3}
\begin{sphinxuseclass}{sd-shadow-sm}
\begin{sphinxuseclass}{sd-card-body}\begin{enumerate}
\sphinxsetlistlabels{\arabic}{enumi}{enumii}{}{.}%
\item {} 
\sphinxAtStartPar
\sphinxguilabel{Load} \sphinxhyphen{} \sphinxguilabel{Load to Masses} 를 클릭합니다.

\end{enumerate}

\end{sphinxuseclass}
\end{sphinxuseclass}
\end{sphinxuseclass}
\end{sphinxuseclass}
\end{sphinxuseclass}
\begin{sphinxuseclass}{sd-card}
\begin{sphinxuseclass}{sd-sphinx-override}
\begin{sphinxuseclass}{sd-mb-3}
\begin{sphinxuseclass}{sd-shadow-sm}
\begin{sphinxuseclass}{sd-card-body}\begin{enumerate}
\sphinxsetlistlabels{\arabic}{enumi}{enumii}{}{.}%
\setcounter{enumi}{1}
\item {} 
\sphinxAtStartPar
생성된 창의 Load Case에 고정하중만 포함되게 변경, 추가, 제거합니다.
설계자에 따라 고정하중의 이름이 바뀔 수 있지만, 대부분의 경우 DL이 포함된 하중은 모두 고정하중에 해당합니다.

\end{enumerate}

\end{sphinxuseclass}
\end{sphinxuseclass}
\end{sphinxuseclass}
\end{sphinxuseclass}
\end{sphinxuseclass}

\paragraph{우발편심 설정}
\label{\detokenize{1_mass_ecc_setting:id4}}
\sphinxstepscope


\subsubsection{주기 확인(탄성설계 모델)}
\label{\detokenize{1_period_check_elastic:id1}}\label{\detokenize{1_period_check_elastic::doc}}
\sphinxAtStartPar
앞서 기술된 단계들을 끝내고 나면, 비교 검증용 탄성설계 모델이 완성됩니다.
본 매뉴얼에서는 세 모델(탄성설계 모델, 레퍼런스 모델, 성능설계 모델)의 주기와 질량참여율을 비교할 것입니다.

\begin{sphinxuseclass}{sd-card}
\begin{sphinxuseclass}{sd-sphinx-override}
\begin{sphinxuseclass}{sd-mb-3}
\begin{sphinxuseclass}{sd-shadow-sm}
\begin{sphinxuseclass}{sd-card-body}
\sphinxAtStartPar
최종 탄성설계 모델을 복사(또는 Save as)하여 새로운 탄성설계 모델을 생성한 후, 파일을 열어줍니다.

\end{sphinxuseclass}
\end{sphinxuseclass}
\end{sphinxuseclass}
\end{sphinxuseclass}
\end{sphinxuseclass}
\sphinxstepscope


\section{Data Conversion Sheet 작성}
\label{\detokenize{2_data_conv_sheets:data-conversion-sheet}}\label{\detokenize{2_data_conv_sheets::doc}}
\sphinxAtStartPar
\DUrole{sd-sphinx-override,sd-badge,sd-bg-success,sd-bg-text-success}{Excel}

\sphinxAtStartPar
Data Conversion Sheet는 성능기반 내진설계에 필요한 모든 정보를 입력할 엑셀 파일입니다.
본 업무절차서에서 소개할 성능기반 내진설계의 모든 과정은 이 엑셀 파일을 기반으로 하며,
따라서 이 파일이 제대로 작성되어야 모델링에서부터 결과 확인까지 오류없이 진행될 수 있습니다.

\sphinxAtStartPar
시트 작성에 앞서, 각 시트의 구성을 아래에 간략하게 소개하겠습니다.
\begin{description}
\item[{ETC}] \leavevmode
\sphinxAtStartPar
철근과 콘크리트의 강도 정보를 입력할 시트

\item[{Materials}] \leavevmode
\sphinxAtStartPar
철근과 콘크리트의 재료 물성치 정보 시트.
참조용(입력 또는 변경 X)

\item[{Nodes / Elements / Nodal Loads / Story Mass / Story Data}] \leavevmode
\sphinxAtStartPar
Midas Gen에서 Import할 정보를 입력할 시트

\item[{Beam Naming / Column Naming / Wall Naming}] \leavevmode
\sphinxAtStartPar
Naming에 필요한 정보를 입력할 시트

\end{description}
\begin{enumerate}
\sphinxsetlistlabels{\Alph}{enumi}{enumii}{}{.}%
\setcounter{enumi}{2}
\item {} \begin{description}
\item[{Beam Properties / G.Column Properties / Wall Properties}] \leavevmode
\sphinxAtStartPar
연결보, 기둥, 벽체의 모든 정보를 입력할 시트

\end{description}

\end{enumerate}
\begin{description}
\item[{Output\_Naming}] \leavevmode
\sphinxAtStartPar
앞에서 입력한 정보들을 바탕으로 이름이 출력되는 시트. 출력용(사용자 입력 X)

\item[{Output\_G.Beam Properties / Output\_E.Beam Properties / Output\_E.Column Properties}] \leavevmode
\sphinxAtStartPar
일반보, 탄성보, 탄성기둥의 모든 정보를 입력할 시트

\item[{Output\_C.Beam Properties / Output\_G.Column Properties / Output\_Wall Properties}] \leavevmode
\sphinxAtStartPar
앞에서 입력한 정보들을 바탕으로 정리된 연결보, 일반기둥, 벽체의 정보가 출력되는 시트

\item[{Results\_C.Beam / Results\_E.Beam / Results\_Wall / Results\_E.Column(개별) / ?????}] \leavevmode
\sphinxAtStartPar
해석결과를 바탕으로 연결보, 탄성보, 벽체, ???의 강도 검토 결과가 출력되는 시트. 출력용(사용자 입력 X)

\end{description}

\begin{sphinxadmonition}{note}{참고:}
\sphinxAtStartPar
레퍼런스 모델은 아래와 같은 이유때문에 중요합니다.
\begin{enumerate}
\sphinxsetlistlabels{\arabic}{enumi}{enumii}{}{.}%
\item {} 
\sphinxAtStartPar
Perform\sphinxhyphen{}3D로 Import하기 위한 최종 모델이기때문에, 이 모델이 정확하지 않으면 성능설계 모델도 부정확하게 모델링됩니다.

\item {} 
\sphinxAtStartPar
탄성설계 모델에서 성능설계 모델로 변환하는 과정에서 변경 사항이 많다면, 레퍼런스 모델이 비교 검증에 가장 중요한 역할을 하는 모델이 됩니다.(?)

\end{enumerate}
\end{sphinxadmonition}

\sphinxAtStartPar
탄성설계 모델과 레퍼런스 모델의 생성 과정은 아래의 단계를 거칩니다.

\sphinxstepscope


\subsection{이름 표기법}
\label{\detokenize{2_naming_rules:id1}}\label{\detokenize{2_naming_rules::doc}}
\sphinxAtStartPar
성능기반 내진설계에서는 부재 각각에 대하여 성능 검증을 해야하기 때문에, 각각의 부재를 확인할 수 있도록 서로 다른 이름을 매겨야 합니다.
Midas Gen과 같이 Perform\sphinxhyphen{}3D도 부재들에게 고유의 ID를 매겨주기는 하지만, 이 번호만으로 부재의 정보를 알아내기는 굉장히 까다롭고 귀찮습니다.
따라서 저희는 수많은 부재들을 쉽게 구별하기 위해, 규칙을 적용하여 각각의 부재에 이름을 매길 것입니다.
부재의 Naming에 앞서, 또 하나의 중요한 정보인 층(Story)의 이름 표기법을 먼저 설명해드리겠습니다.


\subsubsection{층 Naming}
\label{\detokenize{2_naming_rules:naming}}
\sphinxAtStartPar
층의 이름을 매길 때 지켜야 할 규칙은 3가지입니다.
\begin{enumerate}
\sphinxsetlistlabels{\arabic}{enumi}{enumii}{}{.}%
\item {} 
\sphinxAtStartPar
\sphinxstylestrong{3글자 이하} 로 입력할 것.

\item {} 
\sphinxAtStartPar
\sphinxstylestrong{한 글자 이상의 알파벳} 이 포함되게 입력할 것.

\item {} 
\sphinxAtStartPar
\sphinxstylestrong{띄어쓰기, 밑줄(\_), 한글} 을 사용하지 말 것.

\end{enumerate}

\begin{figure}[htbp]
\centering
\capstart

\noindent\sphinxincludegraphics[scale=0.7]{{층_naming_예시}.png}
\caption{\sphinxstyleemphasis{Fig1. 층 Naming 예시}}\label{\detokenize{2_naming_rules:id2}}\end{figure}

\begin{sphinxuseclass}{sd-card}
\begin{sphinxuseclass}{sd-sphinx-override}
\begin{sphinxuseclass}{sd-mb-3}
\begin{sphinxuseclass}{sd-shadow-sm}
\begin{sphinxuseclass}{sd-card-body}
\sphinxAtStartPar
새로 생성한 탄성설계 모델의 단위를 kN, mm로 설정합니다.

\end{sphinxuseclass}
\end{sphinxuseclass}
\end{sphinxuseclass}
\end{sphinxuseclass}
\end{sphinxuseclass}
\begin{sphinxuseclass}{sd-card}
\begin{sphinxuseclass}{sd-sphinx-override}
\begin{sphinxuseclass}{sd-mb-3}
\begin{sphinxuseclass}{sd-shadow-sm}
\begin{sphinxuseclass}{sd-card-body}
\sphinxAtStartPar
최종 탄성설계 모델을 복사(또는 Save as)하여 새로운 탄성설계 모델을 생성한 후, 파일을 열어줍니다.

\end{sphinxuseclass}
\end{sphinxuseclass}
\end{sphinxuseclass}
\end{sphinxuseclass}
\end{sphinxuseclass}

\chapter{Indices and tables}
\label{\detokenize{index:indices-and-tables}}\begin{itemize}
\item {} 
\sphinxAtStartPar
\DUrole{xref,std,std-ref}{genindex}

\item {} 
\sphinxAtStartPar
\DUrole{xref,std,std-ref}{modindex}

\item {} 
\sphinxAtStartPar
\DUrole{xref,std,std-ref}{search}

\end{itemize}



\renewcommand{\indexname}{색인}
\printindex
\end{document}