%% Generated by Sphinx.
\def\sphinxdocclass{report}
\documentclass[a4paper,11pt,korean,openany,oneside]{sphinxmanual}
\ifdefined\pdfpxdimen
   \let\sphinxpxdimen\pdfpxdimen\else\newdimen\sphinxpxdimen
\fi \sphinxpxdimen=.75bp\relax
\ifdefined\pdfimageresolution
    \pdfimageresolution= \numexpr \dimexpr1in\relax/\sphinxpxdimen\relax
\fi
%% let collapsible pdf bookmarks panel have high depth per default
\PassOptionsToPackage{bookmarksdepth=5}{hyperref}
%% turn off hyperref patch of \index as sphinx.xdy xindy module takes care of
%% suitable \hyperpage mark-up, working around hyperref-xindy incompatibility
\PassOptionsToPackage{hyperindex=false}{hyperref}
%% memoir class requires extra handling
\makeatletter\@ifclassloaded{memoir}
{\ifdefined\memhyperindexfalse\memhyperindexfalse\fi}{}\makeatother


\PassOptionsToPackage{warn}{textcomp}

\catcode`^^^^00a0\active\protected\def^^^^00a0{\leavevmode\nobreak\ }
\usepackage{cmap}
\usepackage{fontspec}
\defaultfontfeatures[\rmfamily,\sffamily,\ttfamily]{}
\usepackage{amsmath,amssymb,amstext}
\usepackage{polyglossia}
\setmainlanguage{korean}



\usepackage{kotex}
\usepackage{setspace}
\onehalfspacing
\usepackage[skip=10pt plus1pt]{parskip}
\usepackage[bottom]{footmisc}

\setmainfont{Noto Serif KR}
\setsansfont{Noto Sans KR}
\setmonofont{Noto Sans KR}



\usepackage[Sonny]{fncychap}
\ChNameVar{\Large\normalfont\sffamily}
\ChTitleVar{\Large\normalfont\sffamily}
\usepackage{sphinx}

\fvset{fontsize=\small}
\usepackage{geometry}

\usepackage{fontawesome}

% Include hyperref last.
\usepackage{hyperref}
% Fix anchor placement for figures with captions.
\usepackage{hypcap}% it must be loaded after hyperref.
% Set up styles of URL: it should be placed after hyperref.
\urlstyle{same}


\usepackage{sphinxmessages}
\setcounter{tocdepth}{1}



\title{성능기반 내진설계 업무절차서}
\date{2023년 03월 20일}
\release{}
\author{CNP Dongyang}
\newcommand{\sphinxlogo}{\vbox{}}
\renewcommand{\releasename}{}
\makeindex
\begin{document}

\pagestyle{empty}
\sphinxmaketitle
\pagestyle{plain}
\sphinxtableofcontents
\pagestyle{normal}
\phantomsection\label{\detokenize{pbd_p3d_manual_latex::doc}}


\sphinxstepscope


\chapter{성능기반 내진설계 업무절차서 소개}
\label{\detokenize{0_intro_latex:id1}}\label{\detokenize{0_intro_latex::doc}}
\sphinxAtStartPar
본 업무절차서는 CNP 동양에서의 성능기반 내진설계 수행 절차를 기술한 문서입니다.


\section{필요성}
\label{\detokenize{0_intro_latex:id2}}
\sphinxAtStartPar
성능기반 내진설계 업무절차서는 성능기반 내진설계를 더 \sphinxstylestrong{정확하고, 빠르고, 편리하게} 진행하기 위해 작성되었습니다.


\subsection{모델링 시간 단축}
\label{\detokenize{0_intro_latex:id3}}\begin{enumerate}
\sphinxsetlistlabels{\arabic}{enumi}{enumii}{}{.}%
\item {} 
\sphinxAtStartPar
성능기반 내진설계의 절차는 매우 길고 복잡합니다.
따라서 숙련되지 않은 엔지니어들은 성능기반 내진설계 과정에서 필요한 정보나 절차를 빠뜨리거나 잘못 입력할 수 있습니다.
성능기반 내진설계 업무절차서를 참고함으로써 엔지니어는 \sphinxstylestrong{실수를 줄이고, 실수로 인한 반복 작업을 줄일 수 있습니다.}

\item {} 
\sphinxAtStartPar
성능기반 내진설계에 사용되는 소프트웨어인 Perform\sphinxhyphen{}3D는 숙련되지 않은 사용자가 사용하기에 불편한 점이 많은 소프트웨어입니다.
사용자가 인식하지 못하는 오류들이 발생할 수 있으며, 오류나 예외의 처리가 미흡합니다. 또한 소프트웨어 매뉴얼에도 오류의 처리에 대한 정보가 많지 않습니다.
성능기반 내진설계 업무절차서는 오류를 최대한 방지할 수 있는 모델링 방법을 제시하고 있으며,
오류의 발생이 불가피할 경우, 이를 처리할 수 있는 방법들을 기술하였습니다.
사용자는 이를 통해 \sphinxstylestrong{프로그램 상의 오류를 발견하고 처리하는 불필요한 시간을 줄일 수 있습니다.}

\item {} 
\sphinxAtStartPar
성능기반 내진설계 시간을 증가시키는 가장 큰 요인은 해석 시간입니다.
모델링 방법에 따라 모델링 시간뿐만 아니라 해석 시간도 크게 증가할 수 있기 때문에,
본 업무절차서에서는 \sphinxstylestrong{해석 시간을 최대한 절감할 수 있도록 모델링하는 방법을 기술하였습니다.}

\end{enumerate}


\subsection{학습의 용이함}
\label{\detokenize{0_intro_latex:id4}}
\sphinxAtStartPar
성능기반 내진설계의 절차가 매우 길고 복잡한 만큼, 이를 학습하고 교육하는 것도 복잡하고 시간이 많이 소요됩니다.
또한 Perform\sphinxhyphen{}3D는 Midas Gen, Etabs와 같은 프로그램과는 달리 튜토리얼이나 교육자료뿐만 아니라 유튜브 동영상마저 부족하기 때문에,
스스로 학습이 쉽지 않습니다. 성능기반 내진설계 업무절차서는 \sphinxstylestrong{성능기반 내진설계에 익숙하지 않은 엔지니어들이 학습할 수 있는 자료를 제공할 뿐만 아니라,
선임 엔지니어들이 교육에 쏟아야하는 시간을 절감할 수 있습니다.}


\subsection{지식과 노하우의 공유와 축적}
\label{\detokenize{0_intro_latex:id5}}
\sphinxAtStartPar
본 업무절차서는 CNP 동양의 모든 엔지니어들이 \sphinxstylestrong{성능기반 내진설계에 대한 지식과 노하우를 공유하고 축적하는 플랫폼으로 활용될 수 있습니다.}
성능기반 내진설계에 대해 본 업무절차서와 다른 생각을 갖고 계시거나, 도움이 될 만한 팁, 정보를 공유하고 싶으신 분,
개선할 점이나 오류 등의 의견을 주시고 싶으신 분들이 직접 참여하여 업무절차서를 더 발전시킬 수 있습니다.


\subsection{엔지니어의 건강 증진}
\label{\detokenize{0_intro_latex:id6}}
\sphinxAtStartPar
성능기반 내진설계에서는 사용자가 수동으로 수행해야 할 작업이 매우 많습니다.
수동 작업을 하면 의자에 앉아 작업하는 시간이 늘어나기 때문에, 허리, 시력 등의 육체적 건강에 문제가 생길 수 있습니다.
또한 같은 작업을 계속 반복해야하기 때문에 정신적으로도 피폐해집니다.
성능기반 내진설계 업무절차서는 프로세스를 간소화하고 모델링 시간을 단축시킴으로써, \sphinxstylestrong{엔지니어의 건강에 문제가 될 수 있는 요인들을 줄일 수 있습니다.}


\section{절차}
\label{\detokenize{0_intro_latex:id7}}
\begin{figure}[htbp]
\centering

\noindent\sphinxincludegraphics{{0_프로세스_다이어그램}.png}
\end{figure}


\section{소프트웨어}
\label{\detokenize{0_intro_latex:id8}}
\sphinxAtStartPar
본 성능기반 내진설계 업무절차서에서는 다음의 소프트웨어들을 사용합니다.


\subsection{엑셀}
\label{\detokenize{0_intro_latex:id9}}
\sphinxAtStartPar
엑셀은 대부분의 엔지니어들에게 익숙한 소프트웨어입니다.
특히 많은 데이터를 입력하거나 데이터의 결과값을 확인해야하는 경우, 엑셀은 유용하게 쓰일 수 있습니다.
따라서 본 업무절차서에서는 \sphinxstylestrong{Data Conversion Sheets}라고 하는 엑셀 파일로 모델링 정보를 입력하고 출력할 것입니다.

\sphinxAtStartPar
사용자는 Data Conversion Sheets에 대상 건물에 대한 모든 정보를 입력합니다.
Data Conversion Sheets는 입력된 정보를 이용하여 모델링에 필요한 파라미터, 부재력 등을 계산합니다.
또한 해석결과도 Data Conversion Sheets에서 부분적으로 확인이 가능합니다.


\subsection{PBD\_p3d (In\sphinxhyphen{}house 프로그램)}
\label{\detokenize{0_intro_latex:pbd-p3d-in-house}}
\sphinxAtStartPar
엑셀은 매우 훌륭한 프로그램이지만, 비선형 동적해석(시간이력해석)에서 생성되는 데이터의 양은 엑셀로 처리하기에 벅찹니다.
PBD\_p3d는 방대한 양의 데이터를 처리하는 역할을 하며, 엔


\subsection{key\_macro}
\label{\detokenize{0_intro_latex:key-macro}}
\sphinxAtStartPar
key\_macro(매크로)는 엑셀과 PBD\_p3d

\sphinxAtStartPar
매크로는 엑셀과, PBD\_p3d가 처리할 수 없기 때문에 엔지니어가 직접 처리해야하는 작업들을 대신하는 역할을 합니다.
PBD\_p3d와는 달리 컴퓨터가 작업하는 모습을 화면으로 볼 수 있기 때문에, 그 기능을 직접 확인하실 수 있습니다.
매크로 역시 단점은 있습니다. 분명 사람의 눈과 손보다는 빠르고 정확하게 작동하지만, 방대한 양의 부재를 처리함에 있어서는 빠르다고 말할 수 없습니다.
매크로 실행 중에는 다른 작업을 못하기 때문에, 실행이 끝나기 전까지 PC 사용이 불가능하며, 가끔씩 오류가 생기기도 합니다.
이런 단점들에도 불구하고 단점보다 장점이 더 많은 소프트웨어입니다.


\subsection{Perform\sphinxhyphen{}3D Compound}
\label{\detokenize{0_intro_latex:perform-3d-compound}}

\section{적용기준}
\label{\detokenize{0_intro_latex:id10}}\begin{itemize}
\item {} 
\sphinxAtStartPar
KDS 41 17 00 건축물 내진설계기준(2022), 국토교통부

\item {} 
\sphinxAtStartPar
철근콘크리트 건축구조물의 성능기반 내진설계 지침 (2021), 대한건축학회

\item {} 
\sphinxAtStartPar
철근콘크리트 건축구조물의 성능기반 내진설계를 위한 비선형해석모델 (2021), 대한건축학회

\item {} 
\sphinxAtStartPar
철근콘크리트 건축물 성능기반 내진설계 지침 및 모델링 가이드 (2019), 한국지진공학회

\end{itemize}

\sphinxstepscope


\chapter{레퍼런스 모델 생성}
\label{\detokenize{1_ref_model:id1}}\label{\detokenize{1_ref_model::doc}}
\sphinxAtStartPar
해외 프로젝트를 제외하면 대부분의 프로젝트에서는 탄성설계를 Midas Gen으로 진행합니다. 반면, 저희가 성능설계 모델링에 사용할 프로그램은 Perform\sphinxhyphen{}3D입니다.
따라서 Midas Gen에서 Perform\sphinxhyphen{}3D로 모델링 정보를 변환하는 과정이 필요한데, 두 프로그램은 엄연히 다른 프로그램이기 떄문에 모든 정보를 변환할 수 없습니다.
다만 탄성설계 모델을 최대한 다듬고 수정하여 변환한다면, 최대한 많은 정보들을 변환할 수 있습니다.
또한 모델 정보를 다른 프로그램으로 옮기는 과정이 올바르게 진행되었는지 검증하는 과정도 필요합니다.

\sphinxAtStartPar
이와 같이 \sphinxstylestrong{최종모델(성능기반 내진설계 모델)을 생성하는 과정에서 만들어지는 모든 중간단계의 모델}을 통틀어 레퍼런스 모델이라고 지칭하였습니다.

\begin{figure}[htbp]
\centering
\capstart

\noindent\sphinxincludegraphics[scale=0.55]{{1_레퍼런스_모델_순서}.png}
\caption{\sphinxstyleemphasis{탄성설계 모델 \sphinxhyphen{} 지진파산정 모델 \sphinxhyphen{} 정보변환 모델 \sphinxhyphen{} 주기검토 모델 \sphinxhyphen{} 성능설계 모델}}\label{\detokenize{1_ref_model:id2}}\end{figure}
\subsubsection*{탄성설계 모델}

\sphinxAtStartPar
탄성설계 모델은 \sphinxstylestrong{탄성설계 과정에서 사용된 최종 Midas Gen 모델}입니다.
\subsubsection*{지진파산정 모델}

\sphinxAtStartPar
지진파 산정 모델은 \sphinxstylestrong{지진파 산정 의뢰를 위해 제출해야하는 Midas Gen 모델}입니다.
탄성설계 모델을 성능설계 모델로 변환하는 과정 중에 가장 먼저 생성되는 모델입니다.
\subsubsection*{정보변환 모델}

\sphinxAtStartPar
정보변환 모델은 \sphinxstylestrong{하중을 제외한 모든 정보를 Perform\sphinxhyphen{}3D로 변환하기 위한 Midas Gen 모델}입니다.
하중변환 모델과 정보변환 모델은 Perform\sphinxhyphen{}3D로 Import하기 위한 최종 정보가 포함된 모델이기때문에,
이 두 모델이 정확하지 않으면 성능설계 모델도 부정확하게 모델링될 수 있습니다.

\sphinxAtStartPar
지진파산정 모델에서 몇가지 설정만 변경하여 정보변환 모델을 생성할 수 있습니다.
\subsubsection*{주기검토 모델}

\sphinxAtStartPar
주기검토 모델은 \sphinxstylestrong{탄성설계 모델의 정보가 성능설계 모델에 정확하게 반영되었는지 비교 검증하기 위한 모델}입니다.
주기 외에도 모드형상, 질량참여율, 밑면전단력 등을 비교합니다.
\subsubsection*{성능설계 모델}

\sphinxAtStartPar
성능설계 모델은 \sphinxstylestrong{성능기반 내진설계에 필요한 Perform\sphinxhyphen{}3D 모델}입니다.
하중변환 모델과 정보변환 모델을 기반으로 생성되며, 자세한 내용은 {\hyperref[\detokenize{3_modelling::doc}]{\sphinxcrossref{\DUrole{doc}{모델링}}}} 장에서 다뤄집니다.



\sphinxstepscope


\section{단위 설정}
\label{\detokenize{1_unit_setting:id1}}\label{\detokenize{1_unit_setting::doc}}
\sphinxAtStartPar
모든 모델링에서 가장 먼저 해야할 것은 단위를 설정하는 것입니다. 성능기반 내진설계에서도 마찬가지입니다.
특히 본 매뉴얼에서 소개할 성능기반 내진설계는 여러가지 소프트웨어를 활용하기 때문에, 각각의 소프트웨어마다 단위를 알맞게 변환, 통일하는 것이
매우 중요합니다. 저희는 단위 변환으로 인한 사용자들의 실수와 혼란을 방지하고, 단위 변환에 필요한 프로세스를 줄이기 위해 단위를 통일하여 사용할 것입니다.
저희가 앞으로 사용할 단위는 \(\textbf{kN, mm}\) 입니다.

\begin{sphinxShadowBox}
\sphinxstyletopictitle{What to do}
\begin{enumerate}
\sphinxsetlistlabels{\arabic}{enumi}{enumii}{}{.}%
\item {} 
\sphinxAtStartPar
첫번째로, 지진파산정 모델을 생성하기 위해 최종 탄성설계 모델을 복사(또는 Save as)하여 새로운 모델을 생성합니다. 생성 후 파일을 실행합니다.

\item {} 
\sphinxAtStartPar
새로 생성한 탄성설계 모델의 단위를 \(kN, mm\) 로 설정합니다.

\end{enumerate}
\end{sphinxShadowBox}

\sphinxstepscope


\section{재료 설정}
\label{\detokenize{1_material_setting:id1}}\label{\detokenize{1_material_setting::doc}}
\sphinxAtStartPar
재료 강도는 Midas Gen에서 Perform\sphinxhyphen{}3D로 Import되는 정보가 아니지만,
지진파 산정이나 성능설계 모델과의 비교 검증을 위해 설정해야 합니다.


\subsection{기대강도}
\label{\detokenize{1_material_setting:id2}}
\sphinxAtStartPar
성능기반 내진설계에 사용되는 모델의 재료에는 기대강도를 적용합니다. %
\begin{footnote}[1]\sphinxAtStartFootnote
대한건축학회, 철근콘크리트 건축구조물의 성능기반 내진설계 지침(2021) 4.4
%
\end{footnote}

\begin{figure}[htbp]
\centering
\capstart

\noindent\sphinxincludegraphics{{1_콘크리트_기대강도계수}.png}
\caption{\sphinxstyleemphasis{콘크리트의 기대강도계수}}\label{\detokenize{1_material_setting:id12}}\end{figure}

\sphinxAtStartPar
콘크리트의 기대강도는 콘크리트 강도에 위의 표의 기대강도계수를 곱하여 산정합니다. %
\begin{footnote}[2]\sphinxAtStartFootnote
대한건축학회, 철근콘크리트 건축구조물의 성능기반 내진설계 지침(2021) {[}표 4\sphinxhyphen{}1{]} {[}표 4\sphinxhyphen{}2{]}
%
\end{footnote}
그러나 Midas Gen은 기대강도 대신, 기대강도가 반영된 탄성계수를 입력받습니다.
따라서 아래와 같이 \sphinxstylestrong{기대강도를 반영한 탄성계수를 산정하여 입력}합니다. %
\begin{footnote}[3]\sphinxAtStartFootnote
국토교통부, KDS 14 20 10 콘크리트구조 해석과 설계 원칙(2021), 4.3.3
%
\end{footnote}
\begin{align*}\!\begin{aligned}
E_c &= 8500 \sqrt[3]{기대강도}\\
&= 8500 \sqrt[3]{기대강도계수 \times f_{cm}}\\
\end{aligned}\end{align*}
\begin{sphinxadmonition}{note}{참고:}
\sphinxAtStartPar
기대강도가 반영된 탄성계수는 Data Conversion Sheets의 Materials 시트에서도 확인할 수 있습니다.
\end{sphinxadmonition}


\subsection{포아송비}
\label{\detokenize{1_material_setting:id6}}
\sphinxAtStartPar
포아송비( \(\nu\) )의 경우, 전단탄성계수( \(G\) )를 구하는 식
\begin{equation}\label{equation:1_material_setting:shear_modulus_1}
\begin{split}G = \frac{E}{2(1+\nu)}\end{split}
\end{equation}
\sphinxAtStartPar
를 이용하여 산정할 수 있습니다.

\sphinxAtStartPar
콘크리트의 전단탄성계수는 근사적으로
\begin{equation}\label{equation:1_material_setting:shear_modulus_2}
\begin{split}G_c = 0.4E_c\end{split}
\end{equation}
\sphinxAtStartPar
를 사용하므로 %
\begin{footnote}[4]\sphinxAtStartFootnote
대한건축학회, 철근콘크리트 건축구조물의 성능기반 내진설계를 위한 비선형해석모델(2021) {[}해그림 7\sphinxhyphen{}6{]}
%
\end{footnote}, 성능기반 내진설계에서는 \sphinxstylestrong{두 식에 의해 산정되는 포아송비} \(\textbf{0.25}\)를 사용합니다.

\begin{sphinxShadowBox}
\sphinxstyletopictitle{What to do}
\begin{enumerate}
\sphinxsetlistlabels{\arabic}{enumi}{enumii}{}{.}%
\item {} 
\sphinxAtStartPar
\sphinxguilabel{Properties} \sphinxhyphen{} \sphinxguilabel{Material Properties}을 클릭합니다. 생성된 창에서 변경할 재료를 선택합니다.

\item {} 
\sphinxAtStartPar
콘크리트 정보를 직접 입력하기 위해, Standard를 \sphinxkeyboard{\sphinxupquote{None}}으로 설정합니다.
Modulus of Elasticity에 기대강도를 반영한 탄성계수를 입력합니다.
아래의 그림에서는 C30 콘크리트를 사용하므로, \(E_c = 8500 \sqrt[3]{1.1 \times 30}\)를 계산하여 입력합니다.
Poisson’s Ratio에는 0.25를 입력합니다.

\begin{center}
\noindent\sphinxincludegraphics[scale=0.9]{{1_콘크리트_기대강도_설정}.png}
\end{center}

\item {} 
\sphinxAtStartPar
Thermal Coefficient와 Weight Density는 기존값을 그대로 사용하거나,
기존값이 없는 경우 각각 Thermal Coefficient에는 \(1.0000e-05 \ 1/[C]\),
Weight Density에는 \(2.354e-08 \ kN/mm^3\)을 사용합니다.

\item {} 
\sphinxAtStartPar
\sphinxkeyboard{\sphinxupquote{OK}}를 클릭하여 재료 정보 설정을 완료합니다.

\item {} 
\sphinxAtStartPar
위의 방법으로 탄성설계 모델에 사용된 \sphinxstylestrong{모든 재료}의 콘크리트 강도와 포아송비를 변경합니다.

\end{enumerate}
\end{sphinxShadowBox}

\sphinxstepscope


\section{지하외벽 생성}
\label{\detokenize{1_base_wall:id1}}\label{\detokenize{1_base_wall::doc}}
\sphinxAtStartPar
비선형 해석에서는 지반\sphinxhyphen{}건축물 상호작용을 고려하기 위하여 지하층 구조물을 반드시 고려해야 합니다. %
\begin{footnote}[1]\sphinxAtStartFootnote
대한건축학회, 철근콘크리트 건축구조물의 성능기반 내진설계 지침(2021), 4.6
%
\end{footnote}
탄성설계 모델에 지하층 구조물이 반영되어있는 경우, 지하층 구조물을 그대로 사용하여도 되지만,
지하층 구조물에 대한 정보가 부족하거나 지하층 구조물의 모델링이 까다로운 경우,
설계자와의 협의를 통해 \sphinxstylestrong{간략화된 모델을 적용할 수 있습니다.}

\sphinxAtStartPar
본 업무절차서에서는 지하층 구조물을 간략화하여 모델링하는 경우에 대해 기술하였습니다.


\subsection{지하층 구조물을 간략화하여 모델링하는 경우}
\label{\detokenize{1_base_wall:id3}}
\sphinxAtStartPar
본 업무절차서에서는 지하층의 구조물을 간략화하기 위하여 아래와 같이 모델링합니다.
\begin{enumerate}
\sphinxsetlistlabels{\arabic}{enumi}{enumii}{}{.}%
\item {} 
\sphinxAtStartPar
\sphinxstylestrong{주건물의 외벽으로부터 10m 주위에 지하외벽을 생성}합니다. 다만, 주건물이 지하외벽과 접해있는 경우, 접한 부분은 그대로 모델링합니다.

\item {} 
\sphinxAtStartPar
지하외벽의 두께에 대한 정보가 없는 경우, \sphinxstylestrong{지하외벽의 두께는 450mm}로 모델링합니다.

\item {} 
\sphinxAtStartPar
지하외벽의 강도에 대한 정보가 없는 경우, \sphinxstylestrong{지하외벽의 강도는 기초의 강도와 동일하게 모델링}합니다.

\end{enumerate}

\sphinxAtStartPar
아래에서는 지하외벽을 모델링하는 방법을 예시와 함께 설명합니다.

\begin{sphinxShadowBox}
\sphinxstyletopictitle{What to do}

\sphinxAtStartPar
먼저, 지하외벽의 재료와 두께 정보를 생성합니다.
\begin{enumerate}
\sphinxsetlistlabels{\arabic}{enumi}{enumii}{}{.}%
\item {} 
\sphinxAtStartPar
재료 정보를 생성하는 방법은 {\hyperref[\detokenize{1_material_setting::doc}]{\sphinxcrossref{\DUrole{doc}{재료 설정}}}} 장을 참조합니다.

\sphinxAtStartPar
지하외벽(지하외벽이 없는 경우 기초)의 강도가 C24인 모델의 지하외벽을 생성하는 경우, 아래와 같이 재료 정보를 생성합니다.

\newpage

\begin{center}
\noindent\sphinxincludegraphics[scale=0.9]{{1_지하외벽_강도_설정}.png}
\end{center}

\item {} 
\sphinxAtStartPar
지하외벽의 두께를 설정하기 위해 \sphinxguilabel{Properties} \sphinxhyphen{} \sphinxguilabel{Thickness}를 클릭합니다. 생성된 창에서 \sphinxkeyboard{\sphinxupquote{Add}}를 클릭합니다.

\item {} 
\sphinxAtStartPar
지하외벽의 두께에 대한 정보가 없는 경우, In\sphinxhyphen{}plane \& Out\sphinxhyphen{}of\sphinxhyphen{}plane에 \(450 \ mm\)를 입력합니다.
보통 Thickness ID와 Name에도 두께를 입력하지만, 사용자의 편의대로 입력해도 무방합니다.

\sphinxAtStartPar
설정 후 \sphinxkeyboard{\sphinxupquote{OK}} \sphinxhyphen{} \sphinxkeyboard{\sphinxupquote{Close}}를 클릭합니다.

\newpage

\begin{center}
\noindent\sphinxincludegraphics[scale=0.9]{{1_지하외벽_두께_설정}.png}
\end{center}

\end{enumerate}

\sphinxAtStartPar
지하외벽의 재료와 두께 정보의 입력이 완료되면, 지하외벽을 생성합니다.
\begin{enumerate}
\sphinxsetlistlabels{\arabic}{enumi}{enumii}{}{.}%
\setcounter{enumi}{3}
\item {} 
\sphinxAtStartPar
편리한 지하외벽 모델링을 위하여 모델의 지하층만 \sphinxguilabel{Active}합니다.

\item {} 
\sphinxAtStartPar
지하외벽이 위치할 꼭지점마다 Nodes를 먼저 생성 후, 벽체를 생성할 것입니다.
이를 위해 \sphinxguilabel{Node/Element} \sphinxhyphen{} \sphinxguilabel{Translate Nodes}를 클릭합니다.

\item {} 
\sphinxAtStartPar
주건물의 외벽으로부터 10m 주위에 지하외벽을 생성하는 경우,
건물의 각 꼭지점을 x, y 방향으로 각각 10m가 떨어지는 곳에 복사하는 방식으로 지하외벽의 Nodes를 생성합니다.

\sphinxAtStartPar
지하외벽의 모든 꼭지점이 설정된 모습은 다음과 같습니다.

\newpage

\begin{center}
\noindent\sphinxincludegraphics[scale=0.6]{{1_지하외벽_노드_설정}.png}
\end{center}

\item {} 
\sphinxAtStartPar
생성된 꼭지점을 연결하여 벽체를 생성합니다.

\sphinxAtStartPar
\sphinxguilabel{Node/Element} \sphinxhyphen{} \sphinxguilabel{Create Elements}를 클릭합니다.

\item {} 
\sphinxAtStartPar
Element Type에서 \sphinxkeyboard{\sphinxupquote{Wall}}을 선택한 후, 앞서 만든 지하외벽의 강도와 두께를 각각 Material과 Thickness에서 선택합니다.

\newpage

\begin{center}
\noindent\sphinxincludegraphics[scale=0.9]{{1_지하외벽_벽체_설정}.png}
\end{center}

\item {} 
\sphinxAtStartPar
생성할 벽체의 4개의 꼭지점을 차례대로 클릭하여 벽체를 생성합니다. 모든 지하외벽이 생성된 모습은 다음과 같습니다.

\newpage

\begin{center}
\noindent\sphinxincludegraphics[scale=0.7]{{1_지하외벽_벽체_생성}.png}
\end{center}

\end{enumerate}

\sphinxAtStartPar
정확한 해석을 위해, 생성된 지하외벽을 적정한 간격으로 분할합니다.
\begin{enumerate}
\sphinxsetlistlabels{\arabic}{enumi}{enumii}{}{.}%
\setcounter{enumi}{9}
\item {} 
\sphinxAtStartPar
\sphinxguilabel{Node/Element} \sphinxhyphen{} \sphinxguilabel{Divide Elements}를 클릭합니다.

\item {} 
\sphinxAtStartPar
Element Type에서 \sphinxkeyboard{\sphinxupquote{Wall}}을 선택한 후, Number of Divisions x(수직분할, 층분할)는 \(1\),
Number of Divisions z(수평분할)에는 아래와 같이 적정한 값을 입력합니다.

\newpage

\begin{center}
\noindent\sphinxincludegraphics[scale=0.9]{{1_지하외벽_벽체_분할}.png}
\end{center}

\item {} 
\sphinxAtStartPar
벽체를 선택한 후 \sphinxkeyboard{\sphinxupquote{Apply}}를 클릭하면, 아래와 같이 선택한 벽체가 분할됩니다.

\begin{center}
\noindent\sphinxincludegraphics[scale=0.5]{{1_지하외벽_벽체_분할_2}.png}
\end{center}

\item {} 
\sphinxAtStartPar
위의 방법으로 생성한 \sphinxstylestrong{모든 지하외벽}을 적정한 간격으로 분할합니다.

\end{enumerate}
\end{sphinxShadowBox}

\sphinxstepscope


\section{지점 및 다이어프램 설정}
\label{\detokenize{1_support_setting:id1}}\label{\detokenize{1_support_setting::doc}}

\subsection{지점 설정}
\label{\detokenize{1_support_setting:id2}}
\sphinxAtStartPar
성능기반 내진설계에서는 원칙적으로 기초면 하부가 고정된 모델을 사용하여야 합니다. %
\begin{footnote}[1]\sphinxAtStartFootnote
한국지진공학회, 철근콘크리트 건축물 성능기반 내진설계 지침 및 모델링 가이드(2019) 3.4\sphinxhyphen{}(1)
%
\end{footnote}
그러나 경우에 따라 지하구조 측면의 구속효과를 고려하거나 지표면에서 고정조건을 사용하는 경우도 있으므로, 설계자와의 협의 이 후에 결정합니다.

\sphinxAtStartPar
본 성능기반 내진설계 업무절차서에서는 기초면 하부가 고정된 경우에 대해 모델링합니다.

\begin{sphinxShadowBox}
\sphinxstyletopictitle{What to do}
\begin{enumerate}
\sphinxsetlistlabels{\arabic}{enumi}{enumii}{}{.}%
\item {} 
\sphinxAtStartPar
기초면 하부에 지점을 설정하기에 앞서, 탄성설계 단계에서 설정된 지점과의 혼동을 피하기 위해 탄성설계 단계에서 설정된 지점을 삭제합니다.

\sphinxAtStartPar
\sphinxguilabel{Tree Menu}의 \sphinxguilabel{Works} 탭에서 \sphinxguilabel{Boundaries} \sphinxhyphen{} \sphinxguilabel{Supports}를 선택합니다.

\item {} 
\sphinxAtStartPar
설정되어있는 Supports를 마우스 오른쪽 버튼으로 클릭한 후 \sphinxkeyboard{\sphinxupquote{Delete}}을 눌러 모두 삭제합니다.

\item {} 
\sphinxAtStartPar
기초면 하부를 고정단으로 설정하기 위하여 Midas Gen에서 \sphinxguilabel{Boundary} \sphinxhyphen{} \sphinxguilabel{Define Supports}를 클릭합니다.

\item {} 
\sphinxAtStartPar
모델의 가장 하단(기초면)을 선택합니다.

\begin{center}
\noindent\sphinxincludegraphics[scale=0.9]{{1_기초면_선택}.png}
\end{center}

\item {} 
\sphinxAtStartPar
Dx, Dy, Dz를 체크하고 \sphinxkeyboard{\sphinxupquote{Apply}}를 클릭합니다.

\sphinxAtStartPar
설정이 완료되면, 기초면에 생성한 지점이 표시됩니다.

\begin{center}
\noindent\sphinxincludegraphics[scale=0.9]{{1_supports_설정}.png}
\end{center}

\end{enumerate}
\end{sphinxShadowBox}


\subsection{다이어프램 설정}
\label{\detokenize{1_support_setting:id4}}
\sphinxAtStartPar
비선형해석에서 슬래브의 모델링은 전체 층에 대하여 면내강체로 정의되어야합니다. %
\begin{footnote}[2]\sphinxAtStartFootnote
대한건축학회, 철근콘크리트 건축구조물의 성능기반 내진설계 지침(2021), 4.5\sphinxhyphen{}(1)
%
\end{footnote}
따라서 전체 층이 면내강체(Rigid Diaphragm)로 설정되어있는지 확인하고, 면내강체가 설정되어있지 않은 층에는 면내강체를 설정합니다.
다만 지점조건이나 Ground Level에 따라 다이어프램의 설정 역시 달라질 수 있으므로, 설계자와 협의하여 결정합니다.

\begin{sphinxShadowBox}
\sphinxstyletopictitle{What to do}
\begin{enumerate}
\sphinxsetlistlabels{\arabic}{enumi}{enumii}{}{.}%
\item {} 
\sphinxAtStartPar
Rigid Diaphragm 설정을 위하여 \sphinxguilabel{Structure}\sphinxhyphen{} \sphinxguilabel{Control Data…} \sphinxhyphen{} \sphinxguilabel{Story…}을 클릭합니다.

\item {} 
\sphinxAtStartPar
생성된 창에서 가장 아래층(기초면)을 제외한 모든 층의 \sphinxcode{\sphinxupquote{Floor Diaphragm}}이 Consider로 설정되어있는지 확인합니다.

\begin{center}
\noindent\sphinxincludegraphics[scale=0.9]{{1_diaphragm_설정}.png}
\end{center}

\item {} 
\sphinxAtStartPar
Consider로 설정되어있지 않은 층(Do not Consider로 설정된 층)은 Consider로 변경합니다. 설정 완료 후 \sphinxkeyboard{\sphinxupquote{Close}}를 클릭합니다.

\end{enumerate}
\end{sphinxShadowBox}

\sphinxstepscope


\section{유효강성 설정}
\label{\detokenize{1_stiffness_setting:id1}}\label{\detokenize{1_stiffness_setting::doc}}
\sphinxAtStartPar
성능기반 내진설계에서의 유효강성은 아래의 표%
\begin{footnote}[1]\sphinxAtStartFootnote
대한건축학회, 철근콘크리트 건축구조물의 성능기반 내진설계 지침(2021) {[}표 6\sphinxhyphen{}1{]}
%
\end{footnote} 에 따릅니다.

\begin{figure}[htbp]
\centering
\capstart

\noindent\sphinxincludegraphics{{1_유효강성}.png}
\caption{\sphinxstyleemphasis{비선형모델의 유효강성(대한건축학회)}}\label{\detokenize{1_stiffness_setting:id10}}\end{figure}

\begin{sphinxadmonition}{note}{참고:}
\sphinxAtStartPar
재료강도와 마찬가지로 유효강성도 Midas Gen에서 Perform\sphinxhyphen{}3D로 Import되는 정보가 아니지만,
이 후 생성할 레퍼런스 모델, 성능설계 모델과의 비교 검증을 위해 사용됩니다.
\end{sphinxadmonition}

\sphinxAtStartPar
위의 표를 참고하여 Midas Gen에서 유효강성을 변경합니다. 다만 전단강성 \(GA_W\)의 경우, 계산에 필요한 단면적 \(A_W\)가
유효단면적( \(A_e\) )이 아닌 전체단면적( \(A_g\) )임에 주의해야 합니다.
Midas Gen에서는 유효단면적(\(A_e = \frac{5}{6}A_g\) ; 모든 보의 단면적은 직사각형으로 가정함)을 자동으로 계산하여 사용하므로,
역수인 \(\frac{6}{5}(\approx 1.2)\)를 곱하여 전체단면적을 만들어 사용합니다.


\subsection{연결보}
\label{\detokenize{1_stiffness_setting:id3}}
\sphinxAtStartPar
연결보의 유효강성은 아래의 절차에 따라 변경, 추가합니다.

\begin{sphinxShadowBox}
\sphinxstyletopictitle{What to do}
\begin{enumerate}
\sphinxsetlistlabels{\arabic}{enumi}{enumii}{}{.}%
\item {} 
\sphinxAtStartPar
Midas Gen에서 \sphinxguilabel{Properties} \sphinxhyphen{} \sphinxguilabel{Scale Factor} \sphinxhyphen{} \sphinxguilabel{Section Stiffness Scale Factor}를 클릭합니다.

\item {} 
\sphinxAtStartPar
\sphinxguilabel{Section Stiffness Scale Factor} 창에서 변경, 추가할 연결보의 Section을 선택한 후, Scale Factor를 변경하여 줍니다.
휨강성은 \(0.3EI\)이므로, \(I_{yy}, I_{zz}\)에 각각 \(0.3\)을 입력합니다.

\noindent{\hspace*{\fill}\sphinxincludegraphics[scale=0.6]{{1_c_beam_bending_유효강성}.png}\hspace*{\fill}}

\sphinxAtStartPar
입력 후, \sphinxkeyboard{\sphinxupquote{Add/Replace}} 버튼을 누릅니다.

\item {} 
\sphinxAtStartPar
전단강성은 \(0.04(\frac{l}{h})GA\)이므로, \(A_{sy}, I_{sz}\) 의 값을 변경해야 합니다.
연결보의 길이( \(l\) )와 깊이( \(h\) )를 확인한 후, \(0.04(\frac{l}{h})\)를 계산합니다.
위의 설명과 같이, 계산된 값에 \(1.2\)를 곱합니다.

\begin{center}
\noindent\sphinxincludegraphics[scale=0.6]{{1_c_beam_shear_유효강성}.png}
\end{center}

\begin{sphinxadmonition}{warning}{경고:}
\sphinxAtStartPar
Midas Gen 모델링 과정에서 짧은 벽을 생략하는 경우, 연결보의 길이가 길게 모델링되는 경우가 있습니다.
따라서 \sphinxstylestrong{도면을 확인} 하여 정확한 연결보의 길이를 이용해 계산합니다.
\end{sphinxadmonition}

\item {} 
\sphinxAtStartPar
모든 연결보의 유효강성을 변경, 추가한 후, \sphinxkeyboard{\sphinxupquote{Close}} 버튼을 누릅니다.

\end{enumerate}
\end{sphinxShadowBox}


\subsection{보, 기둥}
\label{\detokenize{1_stiffness_setting:id4}}
\sphinxAtStartPar
연결보와 동일한 방식으로 \sphinxguilabel{Section Stiffness Scale Factor}에서 유효강성을 변경, 추가합니다.


\subsection{벽체}
\label{\detokenize{1_stiffness_setting:id5}}
\sphinxAtStartPar
벽체의 유효강성은 균열이 있는 경우와 없는 경우로 나누어서 고려합니다.
본 성능기반 내진설계 업무절차서는 벽체에 균열이 있다고 가정하고 모델링할 것입니다.

\sphinxAtStartPar
벽체의 유효강성은 아래의 절차에 따라 변경, 추가합니다.

\begin{sphinxShadowBox}
\sphinxstyletopictitle{What to do}
\begin{enumerate}
\sphinxsetlistlabels{\arabic}{enumi}{enumii}{}{.}%
\item {} 
\sphinxAtStartPar
Midas Gen에서 \sphinxguilabel{Properties} \sphinxhyphen{} \sphinxguilabel{Scale Factor} \sphinxhyphen{} \sphinxguilabel{Wall Stiffness Scale Factor}를 클릭합니다.

\item {} 
\sphinxAtStartPar
벽체의 휨강성은 \(0.7EI\) 이므로, Inplane Stiffness Scale Factor의 Bending \& Axial에 \sphinxcode{\sphinxupquote{0.7}}을 입력합니다.

\item {} 
\sphinxAtStartPar
전단강성은 \(0.5GA\) 이지만, 연결보와 마찬가지로 전체단면적의 사용을 위해 \(1.2\)를 곱합니다.
\(0.5 \times 1.2 = 0.6\)이므로, \sphinxcode{\sphinxupquote{0.6}}을 Inplane Stiffness Scale Factor의 Shear에 입력합니다.

\begin{center}
\noindent\sphinxincludegraphics[scale=0.6]{{1_wall_유효강성}.png}
\end{center}

\item {} 
\sphinxAtStartPar
모든 벽체를 선택한 후, \sphinxkeyboard{\sphinxupquote{Apply}} 버튼을 눌러 유효강성을 적용합니다.
지하외벽은 일반벽체와 다른 유효강성을 적용하므로, 지하외벽에는 적용하지 않습니다.

\end{enumerate}
\end{sphinxShadowBox}


\subsection{지하외벽}
\label{\detokenize{1_stiffness_setting:id6}}
\sphinxAtStartPar
지하외벽의 유효강성은 아래의 표에 따라 설정해야합니다. %
\begin{footnote}[2]\sphinxAtStartFootnote
한국지진공학회, 철근콘크리트 건축물 성능기반 내진설계 지침 및 모델링 가이드(2019) {[}표 1.3\sphinxhyphen{}2{]}
%
\end{footnote}

\begin{figure}[htbp]
\centering
\capstart

\noindent\sphinxincludegraphics{{1_유효강성_지하외벽}.png}
\caption{\sphinxstyleemphasis{비선형모델의 유효강성(한국지진공학회)}}\label{\detokenize{1_stiffness_setting:id11}}\end{figure}

\sphinxAtStartPar
지하외벽의 유효강성은 일반 벽체와 동일한 방식으로 \sphinxguilabel{Wall Stiffness Scale Factor}에서 변경, 추가합니다.
표에 따라 Bending \& Axial(휨강성)은 \(0.8\), Shear(전단강성)는 \(0.6\)로 설정한 후, 모든 지하외벽에 적용합니다.

\sphinxstepscope


\section{질량 및 우발편심 설정}
\label{\detokenize{1_mass_ecc_setting:id1}}\label{\detokenize{1_mass_ecc_setting::doc}}

\subsection{질량 설정}
\label{\detokenize{1_mass_ecc_setting:id2}}
\sphinxAtStartPar
비선형 해석에서는 고정하중에 해당되는 질량만 고려됩니다. %
\begin{footnote}[1]\sphinxAtStartFootnote
대한건축학회, 철근콘크리트 건축구조물의 성능기반 내진설계 지침(2021), 4.1\sphinxhyphen{}(4)
%
\end{footnote}
그러나 탄성설계 과정에서는 엔지니어의 판단에 따라 고정하중 뿐만아니라 활하중도 고려되기도 합니다.
따라서 \sphinxstylestrong{같은 조건에서의 성능설계 모델과의 비교} 및 \sphinxstylestrong{고정하중만 고려된 질량의 Import}를 위해 Midas Gen에서 질량을 설정해주어야 합니다.

\begin{sphinxShadowBox}
\sphinxstyletopictitle{What to do}
\begin{enumerate}
\sphinxsetlistlabels{\arabic}{enumi}{enumii}{}{.}%
\item {} 
\sphinxAtStartPar
\sphinxguilabel{Load} \sphinxhyphen{} \sphinxguilabel{Load to Masses} 를 클릭합니다.

\item {} 
\sphinxAtStartPar
생성된 창의 Load Case에 고정하중만 포함되게 변경, 추가, 제거합니다.
설계자에 따라 고정하중의 이름이 바뀔 수 있지만, 대부분의 경우 DL이 포함된 하중은 모두 고정하중에 해당합니다.
지침에 따라 고정하중의 100\%를 적용하여야 하므로, Scale Factor는 1로 설정합니다.

\noindent{\hspace*{\fill}\sphinxincludegraphics[scale=0.6]{{2_질량_고정하중_설정}.png}\hspace*{\fill}}

\end{enumerate}
\end{sphinxShadowBox}

\sphinxAtStartPar
또한 비선형 해석에서는 지반\sphinxhyphen{}건축물 상호작용을 고려하기 위하여 지하층 구조물을 반드시 고려해야 합니다. %
\begin{footnote}[2]\sphinxAtStartFootnote
대한건축학회, 철근콘크리트 건축구조물의 성능기반 내진설계 지침(2021), 4.6\sphinxhyphen{}(1)
%
\end{footnote}
따라서 지하층 구조물의 질량을 모델에 포함시키도록 설정합니다.

\begin{sphinxShadowBox}
\sphinxstyletopictitle{What to do}
\begin{enumerate}
\sphinxsetlistlabels{\arabic}{enumi}{enumii}{}{.}%
\item {} 
\sphinxAtStartPar
\sphinxguilabel{Structure} \sphinxhyphen{} \sphinxguilabel{Contro Data…} \sphinxhyphen{} \sphinxguilabel{Contro Data…}를 클릭합니다.

\item {} 
\sphinxAtStartPar
생성된 창에서 \sphinxguilabel{Consider Mass below Ground Level for Eigenvalue Analysis}를 체크합니다.

\noindent{\hspace*{\fill}\sphinxincludegraphics[scale=0.6]{{2_질량_지하층_설정}.png}\hspace*{\fill}}

\end{enumerate}
\end{sphinxShadowBox}


\subsection{우발편심 설정}
\label{\detokenize{1_mass_ecc_setting:id5}}
\sphinxAtStartPar
성능기반 내진설계를 수행할 건물은 건축물 내진설계기준에 따른 우발편심이 고려되어야 합니다. %
\begin{footnote}[3]\sphinxAtStartFootnote
대한건축학회, 철근콘크리트 건축구조물의 성능기반 내진설계 지침(2021), 8.2.3

\sphinxAtStartPar
국토교통부, KDS 41 17 00 건축물 내진설계기준(2022), 7.2.6.4
%
\end{footnote}
설계자의 판단에 따라 우발편심의 영향을 고려하지 않는 경우도 있지만,
철근콘크리트 건물은 슬래브를 일체화시켜서 타설하므로 격막이 유연하지 않고 강체로 거동하므로 우발 편심을 고려하는 편이 타당합니다.

\sphinxAtStartPar
따라서 본 절차서는 우발편심을 고려하여 성능기반 내진설계를 진행합니다.


\subsubsection{정적 지진하중 생성}
\label{\detokenize{1_mass_ecc_setting:id7}}
\sphinxAtStartPar
우발편심을 적용하기 위해서는 정적 지진하중을 만들어주어야 합니다.

\sphinxAtStartPar
우발편심(Accidental Eccentricity)이란, 편심을 구할 때 기준에서 명확하게 고려되지 않은 요인들의 영향을 보완하는

\sphinxstepscope


\section{지진파산정 모델}
\label{\detokenize{1_seismic_wave_request_model:id1}}\label{\detokenize{1_seismic_wave_request_model::doc}}
\sphinxAtStartPar
앞선 절차가 모두 완료되면, 해당 파일을 이용해 지진파를 산정할 수 있습니다.

\begin{sphinxShadowBox}
\sphinxstyletopictitle{What to do}
\begin{itemize}
\item {} 
\sphinxAtStartPar
앞의 절차를 모두 마친 파일을 저장한 후 지진파 산정을 요청할 때 사용합니다.

\end{itemize}
\end{sphinxShadowBox}

\sphinxstepscope


\section{하중변환 모델}
\label{\detokenize{1_loads_conversion_model:id1}}\label{\detokenize{1_loads_conversion_model::doc}}
\sphinxAtStartPar
앞선 절차가 모두 완료되면, 해당 파일을 이용해 지진파를 산정할 수 있습니다.

\begin{sphinxShadowBox}
\sphinxstyletopictitle{What to do}
\begin{enumerate}
\sphinxsetlistlabels{\arabic}{enumi}{enumii}{}{.}%
\item {} 
\sphinxAtStartPar
하중변환 모델을 생성하기 위해 지진파산정 모델을 복사(또는 Save as)하여 새로운 모델을 생성합니다.
생성 후, 파일을 실행합니다.

\end{enumerate}
\end{sphinxShadowBox}

\sphinxstepscope


\section{주기 확인(탄성설계 모델)}
\label{\detokenize{1_period_check_model:id1}}\label{\detokenize{1_period_check_model::doc}}
\sphinxAtStartPar
앞서 기술된 단계들을 끝내고 나면, 비교 검증용 탄성설계 모델이 완성됩니다.
본 매뉴얼에서는 세 모델(탄성설계 모델, 레퍼런스 모델, 성능설계 모델)의 주기와 질량참여율을 비교할 것입니다.

\sphinxstepscope


\chapter{Data Conversion Sheets 작성}
\label{\detokenize{2_data_conv_sheets:data-conversion-sheets}}\label{\detokenize{2_data_conv_sheets::doc}}
\sphinxAtStartPar
Data Conversion Sheet는 성능기반 내진설계에 필요한 모든 정보를 입력할 엑셀 파일입니다.
본 업무절차서에서 소개할 성능기반 내진설계의 모든 과정은 이 엑셀 파일을 기반으로 하며,
따라서 이 파일이 제대로 작성되어야 모델링에서부터 결과 확인까지 오류없이 진행될 수 있습니다.

\sphinxAtStartPar
시트 작성에 앞서, 각 시트의 구성을 간략하게 소개합니다.
\begin{itemize}
\item {} \begin{description}
\sphinxlineitem{{\hyperref[\detokenize{2_etc::doc}]{\sphinxcrossref{\DUrole{doc}{ETC}}}}}
\sphinxAtStartPar
철근과 콘크리트의 강도 정보를 입력할 시트.

\end{description}

\item {} \begin{description}
\sphinxlineitem{{\hyperref[\detokenize{2_materials::doc}]{\sphinxcrossref{\DUrole{doc}{Materials}}}}}
\sphinxAtStartPar
철근과 콘크리트의 재료 물성치 정보 시트. 참조용.

\end{description}

\item {} \begin{description}
\sphinxlineitem{{\hyperref[\detokenize{2_nodes_elements::doc}]{\sphinxcrossref{\DUrole{doc}{Nodes / Elements}}}} / {\hyperref[\detokenize{2_nodal_loads::doc}]{\sphinxcrossref{\DUrole{doc}{Nodal Loads}}}} / {\hyperref[\detokenize{2_story_mass::doc}]{\sphinxcrossref{\DUrole{doc}{Story Mass / Elements}}}} / Story Data}
\sphinxAtStartPar
Midas Gen에서 Import할 정보를 입력할 시트.

\end{description}

\item {} \begin{description}
\sphinxlineitem{Naming}
\sphinxAtStartPar
Naming에 필요한 정보를 입력할 시트.

\end{description}

\item {} \begin{description}
\sphinxlineitem{{\hyperref[\detokenize{2_c_beam_properties::doc}]{\sphinxcrossref{\DUrole{doc}{C. Beam Properties}}}} / G.Column Properties / Wall Properties}
\sphinxAtStartPar
연결보, 일반기둥, 벽체의 모든 정보를 입력할 시트.

\end{description}

\item {} \begin{description}
\sphinxlineitem{Output\_Naming}
\sphinxAtStartPar
앞에서 입력한 정보들을 바탕으로 이름이 출력되는 시트.

\end{description}

\item {} \begin{description}
\sphinxlineitem{Output\_G.Beam Properties / Output\_E.Beam Properties / Output\_E.Column Properties}
\sphinxAtStartPar
일반보, 탄성보, 탄성기둥의 모든 정보를 입력할 시트.

\end{description}

\item {} \begin{description}
\sphinxlineitem{Output\_C.Beam Properties / Output\_G.Column Properties / Output\_Wall Properties}
\sphinxAtStartPar
앞에서 입력한 정보들을 바탕으로 정리된 연결보, 일반기둥, 벽체의 정보가 출력되는 시트.

\end{description}

\item {} \begin{description}
\sphinxlineitem{Results\_C.Beam / Results\_G.Beam / Results\_E.Beam / Results\_G.Column / Results\_Wall / Results\_E.Column(개별 file)}
\sphinxAtStartPar
해석결과를 바탕으로 연결보, 일반보, 탄성보, 일반기둥, 벽체, 탄성기둥의 강도 검토 결과가 출력되는 시트.

\end{description}

\end{itemize}

\newpage

\begin{sphinxadmonition}{note}{참고:}
\sphinxAtStartPar
Data Conversion Sheets의 셀은 세가지로 분류됩니다.

\noindent{\hspace*{\fill}\sphinxincludegraphics[scale=0.8]{{2_DCS_셀_구분}.png}\hspace*{\fill}}

\sphinxAtStartPar
사용자는 하얀색 셀에 모델링 정보를 입력합니다.
노란색 셀에도 입력이 가능하지만, PBD\_p3d에서 대부분의 내용을 출력해주기 때문에 수정이 필요한 경우에만 입력합니다.
\end{sphinxadmonition}

\sphinxstepscope


\section{이름 표기법}
\label{\detokenize{2_naming_rules:id1}}\label{\detokenize{2_naming_rules::doc}}
\sphinxAtStartPar
성능기반 내진설계에서는 각각의 부재에 대하여 성능 검증을 수행하기 때문에, 개별 부재마다 서로 다른 이름을 매겨 결과를 확인할 수 있도록 해야합니다.
Midas Gen과 같이 Perform\sphinxhyphen{}3D도 부재들에 고유의 ID를 부여하지만, ID는 단순한 숫자이기 때문에 부재의 정보를 알아내는 것이 까다롭고 시간이 많이 소요됩니다.
따라서 본 매뉴얼에서는, 수많은 부재들을 쉽게 구별하기 위해 규칙을 적용하여 각각의 부재에 이름을 매길 것입니다.

\sphinxAtStartPar
부재의 Naming에 앞서, 또 하나의 중요한 정보인 층(Story)의 이름 표기법을 먼저 설명합니다.


\subsection{층 Naming}
\label{\detokenize{2_naming_rules:naming}}
\sphinxAtStartPar
층의 이름을 매길 때 지켜야 할 규칙은 3가지입니다.

\begin{sphinxadmonition}{important}{중요:}\begin{enumerate}
\sphinxsetlistlabels{\arabic}{enumi}{enumii}{}{.}%
\item {} 
\sphinxAtStartPar
\sphinxstylestrong{3글자 이하}로 입력해야 합니다.

\item {} 
\sphinxAtStartPar
\sphinxstylestrong{한 글자 이상의 알파벳}이 포함되어야 합니다.

\item {} 
\sphinxAtStartPar
\sphinxstylestrong{띄어쓰기, 밑줄(\_), 한글}을 사용할 수 없습니다.

\end{enumerate}
\end{sphinxadmonition}

\begin{figure}[htbp]
\centering
\capstart

\noindent\sphinxincludegraphics[scale=0.8]{{2_층_이름_예시}.png}
\caption{\sphinxstyleemphasis{층 이름의 예시}}\label{\detokenize{2_naming_rules:id3}}\end{figure}


\subsection{부재 Naming}
\label{\detokenize{2_naming_rules:id2}}
\sphinxAtStartPar
시트에 입력되는 모든 부재들(벽체, 기둥, 보)은 아래의 구조 형식을 이루고 있어야 합니다.

\begin{figure}[htbp]
\centering
\capstart

\noindent\sphinxincludegraphics[scale=0.4]{{2_부재_이름_구조}.png}
\caption{\sphinxstyleemphasis{부재 이름의 구조}}\label{\detokenize{2_naming_rules:id4}}\end{figure}
\begin{itemize}
\item {} \begin{description}
\sphinxlineitem{부재 이름}
\sphinxAtStartPar
부재 일람표, 도면 등에 표기된 부재의 이름

\end{description}

\item {} \begin{description}
\sphinxlineitem{부재 번호}
\sphinxAtStartPar
건물의 평면 상에 같은 이름의 부재가 여러 개 있는 경우, 그 부재들을 구별해주기 위한 번호.
단, 동일한 부재가 없는 단일 부재이더라도 부재 번호가 존재해야 함.

\end{description}

\item {} \begin{description}
\sphinxlineitem{층}
\sphinxAtStartPar
해당 부재가 위치하고 있는 층.

\end{description}

\end{itemize}

\sphinxAtStartPar
또한 부재 이름 구조뿐만 아니라, 아래의 규칙을 따라야 합니다.

\begin{sphinxadmonition}{important}{중요:}\begin{enumerate}
\sphinxsetlistlabels{\arabic}{enumi}{enumii}{}{.}%
\item {} 
\sphinxAtStartPar
밑줄(\_)은 세 구성요소(부재이름, 부재번호, 층)를 구별해주는 역할을 합니다. 이 외의 \sphinxstylestrong{추가적인 밑줄(\_) 사용은 제한}됩니다.
밑줄(\_)을 제외한 다른 특수문자로 대체하여 사용해주세요.

\item {} 
\sphinxAtStartPar
“부재 이름” 구성요소에는 \sphinxstylestrong{한 글자 이상의 알파벳}이 포함되어야 합니다.

\item {} 
\sphinxAtStartPar
\sphinxstylestrong{한글, 띄어쓰기}를 사용할 수 없습니다.

\end{enumerate}
\end{sphinxadmonition}

\begin{figure}[htbp]
\centering
\capstart

\noindent\sphinxincludegraphics[scale=0.6]{{2_부재_이름_예시}.png}
\caption{\sphinxstyleemphasis{부재 이름의 예시}}\label{\detokenize{2_naming_rules:id5}}\end{figure}

\begin{sphinxShadowBox}
\sphinxstyletopictitle{What to do}
\begin{itemize}
\item {} 
\sphinxAtStartPar
앞서서 생성한 Midas Gen 파일들의 층이름을 위의 규칙에 맞게 변경합니다.

\item {} 
\sphinxAtStartPar
앞으로 입력해야할 부재의 이름을 위의 구조와 규칙에 맞게 설정합니다.

\end{itemize}
\end{sphinxShadowBox}

\sphinxstepscope


\section{ETC}
\label{\detokenize{2_etc:etc}}\label{\detokenize{2_etc::doc}}
\sphinxAtStartPar
ETC 시트에는 해당 건물의 재료 강도를 입력합니다.

\sphinxAtStartPar
해당 건물의 콘크리트와 철근 강도는 보통 구조계산서에서 확인 가능하며,
구조계산서가 없을 시, Midas Gen 모델이나 부재 일람표를 참조할 수 있습니다.

\sphinxAtStartPar
ETC 시트의 작성은 \sphinxstylestrong{예시를 통해 설명}합니다.


\subsection{철근 강도}
\label{\detokenize{2_etc:id1}}
\sphinxAtStartPar
철근 강도는 \sphinxstylestrong{철근 종류에 해당하는 강도를 하나씩 입력하는 방식}으로 입력합니다.

\begin{sphinxadmonition}{note}{Example}

\sphinxAtStartPar
아래의 그림은 실제 구조계산서에 작성되어있는 철근 강도입니다.

\begin{figure}[H]
\centering
\capstart

\noindent\sphinxincludegraphics[scale=0.8]{{2_철근강도_구조계산서}.png}
\caption{\sphinxstyleemphasis{구조계산서에 기입된 철근 강도}}\label{\detokenize{2_etc:id6}}\end{figure}

\sphinxAtStartPar
ETC 시트의 철근 종류(\sphinxcode{\sphinxupquote{Type}}열)에 해당하는 철근 강도를 \sphinxcode{\sphinxupquote{일반용}} 또는 \sphinxcode{\sphinxupquote{내진용}}열에 기입합니다.

\noindent{\hspace*{\fill}\sphinxincludegraphics[scale=0.8]{{2_철근강도_기입}.png}\hspace*{\fill}}
\begin{enumerate}
\sphinxsetlistlabels{\arabic}{enumi}{enumii}{}{.}%
\item {} 
\sphinxAtStartPar
D10 이하의 철근 강도는 SD400이므로, D10, 일반용에 SD400을 입력합니다.

\item {} 
\sphinxAtStartPar
D13 이하의 철근 강도는 SD500이므로, D13, 일반용에 SD500을 입력합니다.

\item {} 
\sphinxAtStartPar
D16 이상의 철근 강도는 SD600이므로, D16부터 D57까지, 일반용에 SD600을 입력합니다.

\item {} 
\sphinxAtStartPar
D16 이상의 전이보, 전이기둥 철근 강도는 SD600S이므로, D16부터 D57까지, 내진용에 SD600S을 입력합니다.

\end{enumerate}

\begin{sphinxadmonition}{warning}{경고:}
\sphinxAtStartPar
사용되지 않는 부재이더라도, (확인 후 작성할 예정)
\end{sphinxadmonition}
\end{sphinxadmonition}


\subsection{콘크리트 강도}
\label{\detokenize{2_etc:id2}}
\sphinxAtStartPar
콘크리트 강도는 \sphinxstylestrong{해당 콘크리트 강도가 적용되는 층의 범위를 입력하는 방식}으로 입력합니다.

\begin{sphinxadmonition}{note}{Example 1}

\sphinxAtStartPar
아래의 그림은 실제 구조계산서에 작성되어있는 콘크리트 강도입니다.

\begin{figure}[H]
\centering

\noindent\sphinxincludegraphics[scale=0.8]{{2_콘크리트강도_구조계산서}.png}
\end{figure}

\sphinxAtStartPar
수직재는 수직재(\sphinxcode{\sphinxupquote{Vertical Member}})열에, 수평재는 수평재(\sphinxcode{\sphinxupquote{Horizontal Member}})열에 입력합니다.

\sphinxAtStartPar
콘크리트 강도를 \sphinxcode{\sphinxupquote{Concrete}}열에 입력하고, 해당 강도가 적용되는 층의 범위를 \sphinxcode{\sphinxupquote{Story(from)}}부터 \sphinxcode{\sphinxupquote{Story(to)}}열까지 입력합니다.

\begin{figure}[H]
\centering

\noindent\sphinxincludegraphics[scale=0.8]{{2_콘크리트강도_기입}.png}
\end{figure}
\begin{enumerate}
\sphinxsetlistlabels{\arabic}{enumi}{enumii}{}{.}%
\item {} 
\sphinxAtStartPar
16F 수직재는 수직재이므로, \sphinxcode{\sphinxupquote{Story(from)}}열에는 16F를 입력합니다.

\item {} 
\sphinxAtStartPar
최상층은 Roof이지만 Roof에는 수직재가 없습니다. 따라서 \sphinxcode{\sphinxupquote{Story(to)}}열에는 Roof의 아래층인 36F를 입력합니다.

\item {} 
\sphinxAtStartPar
해당되는 콘크리트 강도인 C24를 \sphinxcode{\sphinxupquote{Concrete}}열에 입력합니다.

\end{enumerate}
\end{sphinxadmonition}

\begin{sphinxadmonition}{note}{Example 2}

\sphinxAtStartPar
Example 1의 콘크리트 강도를 사용합니다.
\begin{enumerate}
\sphinxsetlistlabels{\arabic}{enumi}{enumii}{}{.}%
\item {} 
\sphinxAtStartPar
6F 수직재는 수직재입니다. 해당층의 보는 층의 바닥에 위치하므로, 6F 보는 층범위에 해당하지않습니다.
따라서 \sphinxcode{\sphinxupquote{Story(from)}}열에는 그 위층 보인 7F를 입력합니다.

\item {} 
\sphinxAtStartPar
11F 수평재는 11F 보입니다. 따라서 \sphinxcode{\sphinxupquote{Story(to)}}열에는 11F를 입력합니다.

\item {} 
\sphinxAtStartPar
해당되는 콘크리트 강도인 C24를 \sphinxcode{\sphinxupquote{Concrete}}열에 입력합니다.

\end{enumerate}
\end{sphinxadmonition}

\begin{sphinxadmonition}{note}{참고:}
\sphinxAtStartPar
콘크리트 강도와 층의 범위의 입력은 특별한 순서에 따르지 않아도 됩니다.
\end{sphinxadmonition}

\begin{sphinxShadowBox}
\sphinxstyletopictitle{What to do}

\sphinxAtStartPar
위의 예시에 따라 철근과 콘크리트 강도를 입력합니다.
\end{sphinxShadowBox}


\subsection{벽체 특수경계요소의 유무}
\label{\detokenize{2_etc:id3}}
\sphinxAtStartPar
\sphinxcode{\sphinxupquote{Acceptable Plastic Hinge Rotation\_Wall}}열은 벽체의 성능설계 허용기준이 저장되어있는 열입니다.
벽체의 허용기준은 특수경계요소의 유무에 따라 달라지기 때문에%
\begin{footnote}[1]\sphinxAtStartFootnote
대한건축학회, 철근콘크리트 건축구조물의 성능기반 내진설계를 위한 비선형해석모델(2021) {[}표 6\sphinxhyphen{}1{]}
%
\end{footnote}, 사용자가 \sphinxstylestrong{벽체의 특수경계요소의 유무를 직접 확인하고 지정}해야합니다.

\begin{sphinxShadowBox}
\sphinxstyletopictitle{What to do}

\sphinxAtStartPar
벽체의 특수경계요소 유무를 확인하고, \sphinxcode{\sphinxupquote{Boundary}}열에서 유무를 선택 또는 입력합니다.
\end{sphinxShadowBox}

\sphinxstepscope


\section{Materials}
\label{\detokenize{2_materials:materials}}\label{\detokenize{2_materials::doc}}
\sphinxAtStartPar
Materials 시트는 철근과 콘크리트의 재료 물성치가 저장된 시트입니다.

\sphinxAtStartPar
이 시트의 내용들은 사용자가 입력 또는 수정하지 않아도 됩니다.

\sphinxstepscope


\section{Nodes / Elements}
\label{\detokenize{2_nodes_elements:nodes-elements}}\label{\detokenize{2_nodes_elements::doc}}
\sphinxAtStartPar
Nodes와 Elements 시트에는 Midas Gen에서 Import할 Nodes와 Elements 정보를 입력합니다.

\begin{sphinxShadowBox}
\sphinxstyletopictitle{What to do}
\begin{itemize}
\item {} 
\sphinxAtStartPar
Midas Gen에서 레퍼런스 모델을 열고, \sphinxguilabel{Nodes Table}과 \sphinxguilabel{Elements Table}의
데이터를 각각 Nodes, Elements 시트에 \sphinxkeyboard{\sphinxupquote{Ctrl}}+ \sphinxkeyboard{\sphinxupquote{C}}, \sphinxkeyboard{\sphinxupquote{Ctrl}}+ \sphinxkeyboard{\sphinxupquote{V}}합니다.

\end{itemize}
\end{sphinxShadowBox}

\begin{sphinxadmonition}{warning}{경고:}\begin{itemize}
\item {} 
\sphinxAtStartPar
기존에 사용하던 시트에 덮어씌우는 경우, 기존 데이터가 모두 지워졌는지 다시 한 번 확인합니다.

\item {} 
\sphinxAtStartPar
Nodes 정보를 Import하지 않더라도, Elements, Nodal Loads, Gages 정보를 Import하려면 \sphinxstylestrong{Nodes 시트가 반드시 작성되어야 합니다.}

\end{itemize}
\end{sphinxadmonition}

\sphinxstepscope


\section{Nodal Loads}
\label{\detokenize{2_nodal_loads:nodal-loads}}\label{\detokenize{2_nodal_loads::doc}}
\sphinxAtStartPar
Perform\sphinxhyphen{}3D에서는 하중을 점(Nodal Loads), 선(Element Loads)의 형식으로만 입력할 수 있습니다.
따라서 Midas Gen의 Floor Loads와 같이 면의 형태로 작용하는 하중은 Nodal Loads, Element Loads로 바꾸어 생성 또는 Import해야합니다.

\sphinxAtStartPar
\DUrole{versionmodified,deprecated}{버전 Beta부터 폐지됨: }기존에는 SDS를 사용하여 Floor Loads를 Nodal Loads로 변환한 후, Nodal Loads를 Import하는 방식을 사용하였습니다.

\sphinxAtStartPar
이러한 Floor Loads 입력의 번거로움을 해결하기 위해, 본 성능기반 내진설계 업무절차서에서는 \sphinxstylestrong{하중 대신 반력}을 입력하는 방식을 사용합니다.
점, 선, 면의 다양한 형태로 입력된 모든 하중에 대한 반력은 수직 부재의 Nodes에서만 발생하기 때문에,
반력을 이용한다면 모든 하중을 Nodal Loads의 형태로 입력할 수 있습니다.


\subsection{Supports 생성}
\label{\detokenize{2_nodal_loads:supports}}
\begin{sphinxShadowBox}
\sphinxstyletopictitle{What to do}
\begin{enumerate}
\sphinxsetlistlabels{\arabic}{enumi}{enumii}{}{.}%
\item {} 
\sphinxAtStartPar
반력은 Supports가 설정된 Nodes에서만 생성됩니다.
따라서 모든 수직 부재에서의 반력을 구하기 위해서는, 모든 수직 부재들에 Supports를 생성해야 합니다.
우선, 모든 수직 부재를 선택하기 위해 \sphinxguilabel{Select Nodes by Identifying…}를 클릭합니다.

\item {} 
\sphinxAtStartPar
생성된 창에서 \sphinxguilabel{Select Type} \sphinxhyphen{} \sphinxguilabel{Material}을 선택한 후, Nodes만 선택하기 위해 \sphinxguilabel{Nodes}만 체크합니다.

\sphinxAtStartPar
Wall, Column과 같은 수직 부재의 재료들만 선택한 후, \sphinxkeyboard{\sphinxupquote{Add}} \sphinxhyphen{} \sphinxkeyboard{\sphinxupquote{Close}}를 클릭합니다.

\item {} 
\sphinxAtStartPar
선택된 수직 부재들에 Supports를 설정해주기 위해 \sphinxguilabel{Boundary} \sphinxhyphen{} \sphinxguilabel{Define Supports}를 클릭합니다.

\sphinxAtStartPar
생성된 창에서 \sphinxguilabel{Dz}만 체크 후 \sphinxkeyboard{\sphinxupquote{Apply}}를 클릭하면, 선택된 모든 수직 부재들에 Supports가 생성됩니다.

\end{enumerate}
\end{sphinxShadowBox}


\subsection{반력 입력}
\label{\detokenize{2_nodal_loads:id1}}\begin{enumerate}
\sphinxsetlistlabels{\arabic}{enumi}{enumii}{}{.}%
\item {} 
\sphinxAtStartPar
Supports의 반력을 확인하기 위해, \sphinxguilabel{Results} \sphinxhyphen{} \sphinxguilabel{Results Table} \sphinxhyphen{} \sphinxguilabel{Reaction}를 클릭합니다.

\sphinxAtStartPar
{\hyperref[\detokenize{2_etc::doc}]{\sphinxcrossref{\DUrole{doc}{하중의 의미}}}}를 다시 한 번 확인하고, 생성된 창에서 DL과 LL에 포함되는 하중을 모두 체크한 후 \sphinxkeyboard{\sphinxupquote{OK}}를 클릭합니다.

\end{enumerate}

\sphinxstepscope


\section{Story Mass}
\label{\detokenize{2_story_mass:story-mass}}\label{\detokenize{2_story_mass::doc}}
\sphinxAtStartPar
Story Mass 시트에는 Midas Gen에서 Import할 Mass 정보를 입력합니다.
다만 우발편심을 고려하는 경우, 앞서 적용하였던 우발편심이 Midas Gen에서 제공하는 질량중심에는 반영되지 않습니다.
따라서 사용자가 직접 우발편심을 질량 중심에 적용한 후, 적용된 질량 중심을 Import해야합니다.


\subsection{Mass 정보 입력}
\label{\detokenize{2_story_mass:mass}}
\begin{sphinxShadowBox}
\sphinxstyletopictitle{What to do}
\begin{enumerate}
\sphinxsetlistlabels{\arabic}{enumi}{enumii}{}{.}%
\item {} 
\sphinxAtStartPar
\sphinxguilabel{Query} \sphinxhyphen{} \sphinxguilabel{Story Mass Table}을 클릭합니다.

\item {} 
\sphinxAtStartPar
생성된 창에서 첫 번째 표(하늘색 셀)만 \sphinxkeyboard{\sphinxupquote{Ctrl}}+ \sphinxkeyboard{\sphinxupquote{C}}하여 Story Mass 시트의 해당 열에 \sphinxkeyboard{\sphinxupquote{Ctrl}}+ \sphinxkeyboard{\sphinxupquote{V}}합니다.

\end{enumerate}
\end{sphinxShadowBox}


\subsection{강성 중심 입력}
\label{\detokenize{2_story_mass:id1}}
\sphinxstepscope


\section{Story Data}
\label{\detokenize{2_story_data:story-data}}\label{\detokenize{2_story_data::doc}}
\sphinxAtStartPar
Story Data 시트는 Midas Gen에서 Import할 층 정보를 입력하는 시트입니다.
대부분의 정보는 자동으로 입력되지만, \sphinxcode{\sphinxupquote{Divide}}열은 사용자가 직접 입력해야 합니다.

\sphinxstepscope


\section{C.Beam Properties}
\label{\detokenize{2_c_beam_properties:c-beam-properties}}\label{\detokenize{2_c_beam_properties::doc}}
\sphinxAtStartPar
C.Beam Properties 시트에는 연결보(Coupling Beam)에 대한 정보를 입력합니다.
연결보의 정보는 부재일람표와 도면 등을 참조합니다.

\sphinxAtStartPar
각 열마다 입력해야할 정보는 다음과 같습니다.

\begin{sphinxadmonition}{note}{참고:}\begin{itemize}
\item {} 
\sphinxAtStartPar
부재는 순서대로(층순, 부재명순 등등) 입력하지 않아도 무방합니다.
사용자 편의에 따라 입력하면, Output\_C.Beam Properties 시트에 자동으로 정렬되어 출력됩니다.

\item {} 
\sphinxAtStartPar
동일한 정보의 부재를 모두 입력할 필요가 없습니다. 아래의 예시와 같이, 해당 행의 정보가 위의 행 정보와 동일하다면, 빈 칸으로 두어도 됩니다.
다만, \sphinxcode{\sphinxupquote{Dimensions}}와 \sphinxcode{\sphinxupquote{Arrangement}} 열은 반드시 모두 입력해야 합니다
(\sphinxcode{\sphinxupquote{Dimensions}} 열은 빈 칸으로 두면 \sphinxcode{\sphinxupquote{Check}}, \sphinxcode{\sphinxupquote{Non\sphinxhyphen{}Check}} 열이 생성되지 않습니다. 또한 \sphinxcode{\sphinxupquote{Arrangement}} 열은 빈 칸으로 두면 0으로 입력됩니다).

\begin{figure}[H]
\centering

\noindent\sphinxincludegraphics{{2_c_beam_예시}.png}
\end{figure}

\end{itemize}
\end{sphinxadmonition}
\begin{itemize}
\item {} \begin{description}
\sphinxlineitem{\sphinxcode{\sphinxupquote{Name}}}
\sphinxAtStartPar
연결보의 이름을 입력합니다.

\end{description}

\item {} \begin{description}
\sphinxlineitem{\sphinxcode{\sphinxupquote{Story}}}
\sphinxAtStartPar
{\hyperref[\detokenize{2_etc::doc}]{\sphinxcrossref{\DUrole{doc}{ETC}}}} 시트에서 콘크리트 강도를 입력했던 방식과 동일하게 해당 연결보가 위치하는 층의 범위를 입력합니다.
해당하는 층이 단일 층인 경우, \sphinxcode{\sphinxupquote{Story(from)}}과 \sphinxcode{\sphinxupquote{Story(to)}}에 동일하게 해당층을 입력합니다.

\begin{sphinxadmonition}{warning}{경고:}
\sphinxAtStartPar
해당 연결보가 불연속적으로 위치하더라도 이를 고려하지 않고 입력합니다.
불연속으로 설치된 층은 이 후의 시트에서 고려하여 입력할 것입니다.
\begin{itemize}
\item {} 
\sphinxAtStartPar
불연속으로 설치된 층의 예시 : 1F, 2F, 4F, 5F (3F에는 설치되어있지 않음)
\sphinxcode{\sphinxupquote{Story(from)}}에는 1F, \sphinxcode{\sphinxupquote{Story(to)}}에는 5F를 입력합니다.

\end{itemize}
\end{sphinxadmonition}

\end{description}

\item {} \begin{description}
\sphinxlineitem{\sphinxcode{\sphinxupquote{Length}}}
\sphinxAtStartPar
연결보의 길이를 입력합니다.
Midas Gen에서는 부재 길이의 부정확성(중심선 사이의 거리만 측정됨),
모델링의 간소화(짧은 벽체의 부재 등)로 인해 길이가 부정확하므로,
정확한 길이의 입력을 위해 도면을 참조하여 입력합니다.

\begin{sphinxadmonition}{warning}{경고:}
\sphinxAtStartPar
같은 이름의 연결보가 여러개 있는 경우, 연결보마다 길이가 다를 수 있습니다.
\sphinxstylestrong{다른 길이의 연결보가 있다면, 그 중 하나의 길이만 선택하여 입력}합니다.
다른 길이의 연결보는 이 후의 시트에서 고려하여 입력할 것입니다.
\end{sphinxadmonition}

\end{description}

\item {} \begin{description}
\sphinxlineitem{\sphinxcode{\sphinxupquote{Dimensions}}}
\sphinxAtStartPar
연결보의 형상을 입력합니다. \sphinxcode{\sphinxupquote{b}}열에는 폭, \sphinxcode{\sphinxupquote{h}}열에는 높이를 각각 입력합니다.

\end{description}

\item {} \begin{description}
\sphinxlineitem{\sphinxcode{\sphinxupquote{Cover}}}
\sphinxAtStartPar
피복두께를 입력합니다. 피복두께가 모두 동일한 경우, 첫번쨰 행에만 피복두께 값만 입력하면 나머지 부재들의 피복두께 값도 자동으로 동일하게 설정됩니다.

\end{description}

\item {} \begin{description}
\sphinxlineitem{\sphinxcode{\sphinxupquote{Rebar}}}
\sphinxAtStartPar
철근 정보를 입력합니다.

\sphinxAtStartPar
\sphinxcode{\sphinxupquote{Type}}, \sphinxcode{\sphinxupquote{배근}}, \sphinxcode{\sphinxupquote{내진상세 여부}}에서 해당되는 철근의 정보를 입력 또는 선택합니다.

\sphinxAtStartPar
\sphinxcode{\sphinxupquote{Main}}, \sphinxcode{\sphinxupquote{Stirrup}}, \sphinxcode{\sphinxupquote{X\sphinxhyphen{}Bracing}} 열에는 주근, 스터럽, X\sphinxhyphen{}브레이싱의 철근 종류을 입력 또는 선택합니다.

\end{description}

\item {} \begin{description}
\sphinxlineitem{\sphinxcode{\sphinxupquote{Arrangement}}}
\sphinxAtStartPar
주근과 스터럽, X\sphinxhyphen{}브레이싱의 개수, 간격, 각도를 입력합니다.
\sphinxcode{\sphinxupquote{Top}} 열에는 아래의 그림을 참조하여 각각의 열에 있는 철근의 개수를 입력합니다.

\begin{figure}[htbp]
\centering

\noindent\sphinxincludegraphics[scale=0.8]{{2_c_beam_properties_section}.png}
\end{figure}

\sphinxAtStartPar
\sphinxcode{\sphinxupquote{Stirrup}} 열에는 스터럽의 개수와 간격을 각각 입력합니다.

\sphinxAtStartPar
\sphinxcode{\sphinxupquote{X}} 열에는 X\sphinxhyphen{}브레이싱의 개수와 각도를 각각 입력합니다.

\end{description}

\item {} \begin{description}
\sphinxlineitem{\sphinxcode{\sphinxupquote{Check}} / \sphinxcode{\sphinxupquote{Non\sphinxhyphen{}Check}}}
\sphinxAtStartPar
\sphinxcode{\sphinxupquote{Check}} / \sphinxcode{\sphinxupquote{Non\sphinxhyphen{}Check}} 열은 해당 연결보가 <철근콘크리트 건축구조물의 성능기반 내진설계 지침(2021)> %
\begin{footnote}[1]\sphinxAtStartFootnote
대한건축학회, 철근콘크리트 건축구조물의 성능기반 내진설계 지침(2021) 6.8.1\sphinxhyphen{}(3)
%
\end{footnote}에 따른 연결보의 범위에 포함되는지 확인합니다.
\begin{itemize}
\item {} \begin{description}
\sphinxlineitem{연결보의 적용범위에 포함되는 경우 (\(1.5 \leq l/h < 4.0\))}
\sphinxAtStartPar
\sphinxcode{\sphinxupquote{1.5 ≤ L/D < 4.0}} 열에 \sphinxstylestrong{Coupling Beam}으로 표시됩니다.
6.8장의 연결보 모델링 방법을 문제없이 적용할 수 있습니다.

\end{description}

\item {} \begin{description}
\sphinxlineitem{연결보의 적용범위에 포함되지 않는 경우 (\(l/h \geq 4.0\))}
\sphinxAtStartPar
\sphinxcode{\sphinxupquote{1.5 ≤ L/D < 4.0}} 열에 \sphinxstylestrong{General Beam}으로 표시됩니다.
6.4장의 보 모델링 방법에 따릅니다. %
\begin{footnote}[2]\sphinxAtStartFootnote
대한건축학회, 철근콘크리트 건축구조물의 성능기반 내진설계 지침(2021) 6.8.1\sphinxhyphen{}(4)
%
\end{footnote}

\end{description}

\item {} \begin{description}
\sphinxlineitem{연결보의 적용범위에 포함되지 않는 경우 (\(l/h <> 1.5\))}
\sphinxAtStartPar
\sphinxcode{\sphinxupquote{Check}} / \sphinxcode{\sphinxupquote{Non\sphinxhyphen{}Check}} 열은 아래와 같이 항목을 선택하여 변경이 가능합니다.

\begin{figure}[htbp]
\centering

\noindent\sphinxincludegraphics{{2_c_beam_properties_check_선택}.png}
\end{figure}
\begin{itemize}
\item {} 
\sphinxAtStartPar
연결보의 형상비가 \(1.5\) 이상이 되도록 부재의 크기를 변경합니다. \sphinxcode{\sphinxupquote{Check}} 열을

\end{itemize}

\end{description}

\end{itemize}

\end{description}

\end{itemize}

\begin{sphinxShadowBox}
\sphinxstyletopictitle{What to do}
\begin{enumerate}
\sphinxsetlistlabels{\arabic}{enumi}{enumii}{}{.}%
\item {} 
\sphinxAtStartPar
위의 정보를 참조하여 연결보의 정보를 입력합니다.

\item {} 
\sphinxAtStartPar
입력이 완료되면 \sphinxcode{\sphinxupquote{Check}}

\end{enumerate}
\end{sphinxShadowBox}

\sphinxstepscope


\section{G.Column Properties}
\label{\detokenize{2_g_column_properties:g-column-properties}}\label{\detokenize{2_g_column_properties::doc}}
\sphinxAtStartPar
G.Column Properties 시트에는 일반기둥(General Column)에 대한 정보를 입력합니다.

\sphinxAtStartPar
각 열마다 입력해야할 정보는 다음과 같습니다.
\begin{itemize}
\item {} \begin{description}
\sphinxlineitem{\sphinxcode{\sphinxupquote{Name}}, \sphinxcode{\sphinxupquote{Story}}, \sphinxcode{\sphinxupquote{Dimensions}}, \sphinxcode{\sphinxupquote{Cover}}}
\sphinxAtStartPar
C.Beam Properties 시트와 동일한 방식으로 입력합니다.

\begin{figure}[htbp]
\centering

\noindent\sphinxincludegraphics{{2_c_beam_예시}.png}
\end{figure}

\end{description}

\item {} \begin{description}
\sphinxlineitem{\sphinxcode{\sphinxupquote{Rebar}}}
\sphinxAtStartPar
철근 정보를 입력합니다.

\sphinxAtStartPar
\sphinxcode{\sphinxupquote{내진상세 여부}} 열에는 해당 철근의 내진상세 여부를 입력 또는 선택합니다.

\sphinxAtStartPar
\sphinxcode{\sphinxupquote{Main}}과 \sphinxcode{\sphinxupquote{Hoop}}의 첫번째 열에는 각각 주근과 후프근이 일반용 철근인지 내진용 철근인지 입력 또는 선택합니다.

\sphinxAtStartPar
\sphinxcode{\sphinxupquote{Main}}과 \sphinxcode{\sphinxupquote{Hoop}}의 두번째 열에는 각각 주근과 후프근의 철근 종류를 입력 또는 선택합니다.

\end{description}

\item {} \begin{description}
\sphinxlineitem{\sphinxcode{\sphinxupquote{Arrangement}}}
\sphinxAtStartPar
주근과 후프근의 개수와 간격을 입력합니다.
\sphinxcode{\sphinxupquote{Layer1}} 열에는 아래의 그림을 참조하여 가장 바깥 레이어의 철근 개수와 행의 개수를 입력합니다.

\sphinxAtStartPar
일반기둥의 배근이 2단으로 되어있는 경우, 안쪽 레이어의 철근 개수와 행의 개수를 \sphinxcode{\sphinxupquote{Layer2}} 열에 입력합니다.

\begin{figure}[htbp]
\centering

\noindent\sphinxincludegraphics[scale=0.6]{{2_g_column_properties_section}.png}
\end{figure}

\sphinxAtStartPar
\sphinxcode{\sphinxupquote{Hoop}}의 \sphinxcode{\sphinxupquote{X}}, \sphinxcode{\sphinxupquote{Y}} 열에는 각각  X\sphinxhyphen{}브레이싱의 개수와 각도를 각각 입력합니다.

\begin{figure}[htbp]
\centering

\noindent\sphinxincludegraphics[scale=0.6]{{2_g_column_properties_section_2}.png}
\end{figure}

\end{description}

\end{itemize}

\begin{sphinxShadowBox}
\sphinxstyletopictitle{What to do}

\sphinxAtStartPar
위의 정보를 참조하여 연결보의 정보를 입력합니다.
\end{sphinxShadowBox}

\sphinxstepscope


\section{Output\_G.Beam Properties / Output\_E.Beam Properties}
\label{\detokenize{2_output_e_g_beam_properties:output-g-beam-properties-output-e-beam-properties}}\label{\detokenize{2_output_e_g_beam_properties::doc}}
\sphinxAtStartPar
Output\_G.Beam Properties 시트에는 일반보(General Beam)에 대한 정보를 입력합니다.
마찬가지로 Output\_E.Beam Properties 시트에는 탄성보(Elastic Beam)에 대한 정보를 입력합니다.
일반보 또는 탄성보가 없을 시, 입력하지 않으셔도 됩니다.

\sphinxAtStartPar
해당 보를 일반보로 모델링할지, 탄성보로 모델링할지의 여부는 설계자와의 협의, 전문가의 자문을 통해 결정합니다. %
\begin{footnote}[1]\sphinxAtStartFootnote
대한건축학회, 철근콘크리트 건축구조물의 성능기반 내진설계 지침(2021) 4.9\sphinxhyphen{}(4)
%
\end{footnote}

\begin{sphinxadmonition}{note}{참고:}
\sphinxAtStartPar
전이구조의 경우, 힘지배거동으로 설계합니다. %
\begin{footnote}[2]\sphinxAtStartFootnote
대한건축학회, 철근콘크리트 건축구조물의 성능기반 내진설계 지침(2021) {[}해표 4\sphinxhyphen{}1{]}
%
\end{footnote}
또한 힘지배거동으로 설계하는 부재는 선형탄성으로 모델링하여야 합니다. {\color{red}\bfseries{}{[}\#{]}\_}
따라서 \sphinxstylestrong{전이보(전이구조)의 경우, 탄성보(선형탄성 부재)로 모델링}합니다.
\end{sphinxadmonition}

\sphinxAtStartPar
입력 방식은 {\hyperref[\detokenize{2_c_beam_properties::doc}]{\sphinxcrossref{\DUrole{doc}{C. Beam Properties}}}}에 연결보를 입력한 방식과 동일합니다.

\begin{sphinxadmonition}{warning}{경고:}
\sphinxAtStartPar
연결보 정보를 C.Beam Properties 시트에 입력할 때와 달리, 사용자가 입력한 순서에 따라
순서를 자동으로 정렬하지 않습니다.
\end{sphinxadmonition}

\begin{sphinxShadowBox}
\sphinxstyletopictitle{What to do}

\sphinxAtStartPar
위의 정보를 참조하여 일반보와 탄성보의 정보를 입력합니다.
\end{sphinxShadowBox}

\sphinxstepscope


\section{Wall Properties}
\label{\detokenize{2_wall_properties:wall-properties}}\label{\detokenize{2_wall_properties::doc}}
\sphinxAtStartPar
Wall Properties 시트에는 벽체에 대한 정보를 입력합니다.

\sphinxAtStartPar
각 열마다 입력해야할 정보는 다음과 같습니다.

\begin{sphinxadmonition}{note}{참고:}\begin{itemize}
\item {} 
\sphinxAtStartPar
부재는 순서대로(층순, 부재명순 등등) 입력하지 않아도 무방합니다.(???)
사용자 편의에 따라 입력하면, Output\_C.Beam Properties 시트에 자동으로 정렬되어 출력됩니다.

\item {} 
\sphinxAtStartPar
동일한 정보의 부재를 모두 입력할 필요가 없습니다. 아래의 예시와 같이, 해당 행의 정보가 위의 행 정보와 동일하다면, 빈 칸으로 두어도 됩니다.
다만, \sphinxcode{\sphinxupquote{Arrangement}} 열은 반드시 모두 입력해야 합니다(빈 칸으로 두면 0으로 입력됩니다).

\begin{figure}[H]
\centering

\noindent\sphinxincludegraphics{{2_c_beam_예시}.png}
\end{figure}

\end{itemize}
\end{sphinxadmonition}
\begin{itemize}
\item {} \begin{description}
\sphinxlineitem{\sphinxcode{\sphinxupquote{Name}}}\begin{quote}

\sphinxAtStartPar
벽체의 이름을 입력합니다.
\end{quote}

\begin{sphinxadmonition}{warning}{경고:}
\sphinxAtStartPar
벽체의 이름을 입력할 때마다 1줄의 공백을 만들어 주어야 합니다.  부재를 입력할 시,
\end{sphinxadmonition}

\end{description}

\item {} \begin{description}
\sphinxlineitem{\sphinxcode{\sphinxupquote{Story}}}
\sphinxAtStartPar
ETC 시트에서 콘크리트 강도를 입력했던 방식과 동일하게 해당 벽체가 위치하는 층의 범위를 입력합니다.
해당하는 층이 단일 층인 경우, \sphinxcode{\sphinxupquote{Story(from)}}과 \sphinxcode{\sphinxupquote{Story(to)}}에 동일하게 해당층을 입력합니다.

\begin{sphinxadmonition}{note}{참고:}
\sphinxAtStartPar
철근콘크리트 전단벽식 건물을 포함해 많은 공동주택에서는 벽체의 양이 많아 Wall Properties 시트를 입력하는 시간이 많이 소요됩니다.
따라서 본 업무절차서에서는 설계자가 제공한 벽체리스트 정보를 활용하여 입력 시간을 줄일 수 있도록 하였습니다.
\begin{itemize}
\item {} 
\sphinxAtStartPar
불연속으로 설치된 층의 예시 : 1F, 2F, 4F, 5F (3F에는 설치되어있지 않음)
\sphinxcode{\sphinxupquote{Story(from)}}에는 1F, \sphinxcode{\sphinxupquote{Story(to)}}에는 5F를 입력합니다.

\end{itemize}
\end{sphinxadmonition}

\end{description}

\item {} \begin{description}
\sphinxlineitem{\sphinxcode{\sphinxupquote{Thickness}}}
\sphinxAtStartPar
벽체의 두께를 입력합니다.

\end{description}

\item {} \begin{description}
\sphinxlineitem{\sphinxcode{\sphinxupquote{Rebar}}}
\sphinxAtStartPar
철근 정보를 입력합니다.

\begin{sphinxadmonition}{note}{참고:}
\sphinxAtStartPar
사용자의 편의에 따라 아래와 같이 철근의 Type과 Spacing을 나누지 않고 입력하여도 무방합니다.
\end{sphinxadmonition}

\end{description}

\item {} \begin{description}
\sphinxlineitem{\sphinxcode{\sphinxupquote{Lo}}}
\sphinxAtStartPar
벽체의 전체 길이를 입력합니다.
연결보와 마찬가지로 Midas Gen에서는 부재 길이의 부정확성(중심선 사이의 거리만 측정됨),
모델링의 간소화(짧은 벽체의 부재 등)로 인해 길이가 부정확하므로,
정확한 길이의 입력을 위해 도면을 참조하여 입력합니다.

\end{description}

\item {} \begin{description}
\sphinxlineitem{\sphinxcode{\sphinxupquote{Le}}}
\sphinxAtStartPar
벽체의 분할된 길이를 입력합니다.
연결보와 마찬가지로 Midas Gen에서는 부재 길이의 부정확성(중심선 사이의 거리만 측정됨),
모델링의 간소화(짧은 벽체의 부재 등)로 인해 길이가 부정확하므로,
정확한 길이의 입력을 위해 도면을 참조하여 입력합니다.

\end{description}

\end{itemize}

\begin{sphinxadmonition}{note}{참고:}
\sphinxAtStartPar
탄성설계에서 작성된 Wall List를 활용하면 Wall Properties 시트를 더 빠르게 작성할 수 있습니다.
따라서 설계자에게 Wall List를 엑셀 형태로 요청하는 것이 좋습니다.
\end{sphinxadmonition}

\begin{sphinxShadowBox}
\sphinxstyletopictitle{What to do}

\sphinxAtStartPar
위의 정보를 참조하여 연결보의 정보를 입력합니다.
\end{sphinxShadowBox}

\sphinxstepscope


\chapter{모델링}
\label{\detokenize{3_modelling:id1}}\label{\detokenize{3_modelling::doc}}
\sphinxAtStartPar
비선형 해석을 수행하기 위하여 Perform\sphinxhyphen{}3D에서 성능설계 모델을 생성합니다.

\sphinxstepscope


\section{탄성설계 모델 Import}
\label{\detokenize{3_import:import}}\label{\detokenize{3_import::doc}}
\sphinxAtStartPar
성능설계 모델은 Perform\sphinxhyphen{}3D로 직접 모델링할 수 있지만,
빠르고 편리한 모델링을 위하여 레퍼런스 모델과 Data Conversion Sheets의 정보를 Import하여 모델링할 수 있습니다.


\subsection{파일 변환 (csv 파일 생성)}
\label{\detokenize{3_import:csv}}
\sphinxAtStartPar
첫 번째 절차는 Midas Gen 모델의 정보를 Perform\sphinxhyphen{}3D에서 읽어들일 수 있는 방식으로 변환하는 것입니다.
변환이 가능한 정보는 {\hyperref[\detokenize{1_material_setting::doc}]{\sphinxcrossref{\DUrole{doc}{Data Conversion Sheets 작성}}}} 장에서 모두 입력되었으므로,
Data Conversion Sheets를 Perform\sphinxhyphen{}3D에서 읽을 수 있는 파일 형식인 \sphinxcode{\sphinxupquote{.csv}}으로 변환합니다.

\begin{sphinxShadowBox}
\sphinxstyletopictitle{What to do}
\begin{enumerate}
\sphinxsetlistlabels{\arabic}{enumi}{enumii}{}{.}%
\item {} 
\sphinxAtStartPar
PBD\_p3d를 실행합니다.

\item {} 
\sphinxAtStartPar
Data Conversion (Excel Sheets)에 Data Conversion Sheets의 경로를 입력합니다.

\item {} 
\sphinxAtStartPar
Import에서 Import를 원하는 항목들을 체크합니다.
Nodal Loads를 Import하는 경우, Dead Load Name과 Live Load Name에 각각 Midas Gen에서 사용하였던 고정하중, 활하중 이름을 입력합니다.

\begin{center}
\noindent\sphinxincludegraphics[scale=0.8]{{3_import_csv_설정}.png}
\end{center}

\item {} 
\sphinxAtStartPar
Import를 클릭합니다. Import가 완료되면 아래의 상태창에 \sphinxcode{\sphinxupquote{Completed!}}가 표시됩니다.
또한 Data Conversion Sheets가 위치하는 경로에 아래와 같이 선택한 항목들의 \sphinxcode{\sphinxupquote{.csv}} 형식 파일이 생성됩니다.

\begin{center}
\noindent\sphinxincludegraphics[scale=0.8]{{3_import_csv_생성}.png}
\end{center}

\end{enumerate}
\end{sphinxShadowBox}


\subsection{Perform\sphinxhyphen{}3D 실행}
\label{\detokenize{3_import:perform-3d}}
\sphinxAtStartPar
Perform\sphinxhyphen{}3D로 새 성능설계 모델 파일을 만들고, 앞서 생성한 csv 파일을 Import합니다. \sphinxcode{\sphinxupquote{Materials}}
새 성능설계 모델 파일은 새 파일을 생성하여 성능설계 모델의 모델링을 시작할 수도 있지만,

\begin{sphinxadmonition}{note}{참고:}
\sphinxAtStartPar
Perform\sphinxhyphen{}3D는 Midas Gen과 같이 단일 파일( \sphinxcode{\sphinxupquote{.mgb}} )로 생성 또는 저장되는 것이 아니라, 폴더 형태로 생성 또는 저장됩니다.
따라서 Perform\sphinxhyphen{}3D 파일을 Load하는 경우, 파일이 아닌 폴더를 선택하여 Load합니다.
\end{sphinxadmonition}

\begin{sphinxShadowBox}
\sphinxstyletopictitle{What to do}
\begin{enumerate}
\sphinxsetlistlabels{\arabic}{enumi}{enumii}{}{.}%
\item {} 
\sphinxAtStartPar
Perform\sphinxhyphen{}3D를 실행한 후, \sphinxkeyboard{\sphinxupquote{Start a New Structure}}를 클릭합니다.

\begin{center}
\noindent\sphinxincludegraphics[scale=0.9]{{3_p3d_실행}.png}
\end{center}

\newpage

\item {} 
\sphinxAtStartPar
Structure Name(파일명), Location of STRUCTURES folder(파일 경로), Structure Description(파일 설명) 등을 입력합니다.
파일명은 \sphinxstylestrong{영문 또는 숫자로, 띄어쓰기 없이} 입력해야 합니다.

\begin{center}
\noindent\sphinxincludegraphics[scale=0.8]{{3_p3d_실행_2}.png}
\end{center}

\sphinxAtStartPar
단위는 {\hyperref[\detokenize{1_unit_setting::doc}]{\sphinxcrossref{\DUrole{doc}{단위 설정}}}} 장에 따라 \(kN, mm\)로 설정합니다.

\sphinxAtStartPar
Minimum spacing between nodes는 \(50 mm\)로 설정합니다.

\sphinxAtStartPar
설정이 완료되면 \sphinxkeyboard{\sphinxupquote{OK}}를 클릭합니다.

\end{enumerate}
\end{sphinxShadowBox}

\newpage


\subsection{Import Nodes}
\label{\detokenize{3_import:import-nodes}}
\begin{sphinxShadowBox}
\sphinxstyletopictitle{What to do}
\begin{enumerate}
\sphinxsetlistlabels{\arabic}{enumi}{enumii}{}{.}%
\item {} 
\sphinxAtStartPar
Nodes를 Import하기 위해 \sphinxguilabel{Import/Export Structure Data}을 클릭하고, 생성된 창에서 \sphinxkeyboard{\sphinxupquote{Import}}를 클릭합니다.

\item {} 
\sphinxAtStartPar
\sphinxguilabel{Nodes Only} 탭을 클릭한 뒤, Specify name of text file에 앞서 생성한 \sphinxcode{\sphinxupquote{Node.csv}} 파일의 경로를 입력합니다.

\begin{center}
\noindent\sphinxincludegraphics[scale=0.9]{{3_import_nodes_설정}.png}
\end{center}

\newpage

\item {} 
\sphinxAtStartPar
입력 후 \sphinxkeyboard{\sphinxupquote{Test}} \sphinxhyphen{} \sphinxkeyboard{\sphinxupquote{확인}}을 클릭하면 아래와 같이 Import할 Nodes가 표시됩니다.

\begin{center}
\noindent\sphinxincludegraphics[scale=0.8]{{3_import_nodes_완료}.png}
\end{center}

\sphinxAtStartPar
\sphinxkeyboard{\sphinxupquote{OK}}를 클릭하여 Import를 완료합니다.

\end{enumerate}
\end{sphinxShadowBox}


\subsection{Import Masses}
\label{\detokenize{3_import:import-masses}}
\begin{sphinxShadowBox}
\sphinxstyletopictitle{What to do}
\begin{enumerate}
\sphinxsetlistlabels{\arabic}{enumi}{enumii}{}{.}%
\item {} 
\sphinxAtStartPar
\sphinxguilabel{Nodes}를 클릭하고 생성된 창에서 \sphinxguilabel{Masses} 탭을 클릭합니다.
Mass Pattern을 만들기 위해 \sphinxkeyboard{\sphinxupquote{New}}를 클릭합니다.

\item {} 
\sphinxAtStartPar
Enter pattern name에 \sphinxcode{\sphinxupquote{Mass}}(또는 사용자가 원하는 이름)를 입력한 후, \sphinxkeyboard{\sphinxupquote{OK}}를 눌러 Mass Patter 생성을 완료합니다.

\begin{center}
\noindent\sphinxincludegraphics[scale=0.9]{{3_import_masses_패턴생성}.png}
\end{center}

\item {} 
\sphinxAtStartPar
Masses를 Import하기 위해 \sphinxguilabel{Import/Export Structure Data}을 클릭하고, 생성된 창에서 \sphinxkeyboard{\sphinxupquote{Import}}를 클릭합니다.

\item {} 
\sphinxAtStartPar
\sphinxguilabel{Masses} 탭을 클릭한 뒤, Choose mass pattern에서 방금 생성한(또는 사용자가 원하는) Mass Pattern을 선택합니다.
Specify name of text file에 앞서 생성한 \sphinxcode{\sphinxupquote{Mass.csv}} 파일의 경로를 입력합니다.

\newpage

\begin{center}
\noindent\sphinxincludegraphics[scale=0.9]{{3_import_masses_설정}.png}
\end{center}

\newpage

\item {} 
\sphinxAtStartPar
입력 후 \sphinxkeyboard{\sphinxupquote{Test}} \sphinxhyphen{} \sphinxkeyboard{\sphinxupquote{확인}}을 클릭하면 아래와 같이 Import할 Nodes가 표시됩니다.

\begin{center}
\noindent\sphinxincludegraphics[scale=0.8]{{3_import_masses_완료}.png}
\end{center}

\sphinxAtStartPar
\sphinxkeyboard{\sphinxupquote{OK}}를 클릭하여 Import를 완료합니다.

\end{enumerate}
\end{sphinxShadowBox}

\newpage


\subsection{Import Elements}
\label{\detokenize{3_import:import-elements}}
\begin{sphinxShadowBox}
\sphinxstyletopictitle{What to do}
\begin{enumerate}
\sphinxsetlistlabels{\arabic}{enumi}{enumii}{}{.}%
\item {} 
\sphinxAtStartPar
\sphinxguilabel{Elements}를 클릭하고, Elements Group을 만들기 위해 생성된 창에서 \sphinxkeyboard{\sphinxupquote{New}}를 클릭합니다.

\item {} 
\sphinxAtStartPar
먼저 연결보를 Import하기 위해 연결보(Coupling Beam) 그룹을 생성합니다.
Element Type에서 \sphinxcode{\sphinxupquote{Beam}}을 선택한 후,
Group Name에 \sphinxcode{\sphinxupquote{C.Beam}}(또는 사용자가 원하는 이름)을 입력합니다.

\begin{center}
\noindent\sphinxincludegraphics[scale=0.9]{{3_import_elements_그룹생성}.png}
\end{center}

\sphinxAtStartPar
\sphinxkeyboard{\sphinxupquote{OK}}를 눌러 연결보 그룹 생성을 완료합니다.

\item {} 
\sphinxAtStartPar
같은 방법으로 연결보 외의 Import할 Elements와 Gages 그룹도 생성합니다.
\begin{quote}\begin{description}
\sphinxlineitem{연결보}
\sphinxAtStartPar
Element Type: \sphinxcode{\sphinxupquote{Beam}} / Group Name: \sphinxcode{\sphinxupquote{C.Beam}}

\sphinxlineitem{일반보}
\sphinxAtStartPar
Element Type: \sphinxcode{\sphinxupquote{Beam}} / Group Name: \sphinxcode{\sphinxupquote{G.Beam}}

\sphinxlineitem{탄성보}
\sphinxAtStartPar
Element Type: \sphinxcode{\sphinxupquote{Beam}} / Group Name: \sphinxcode{\sphinxupquote{E.Beam}}

\sphinxlineitem{일반기둥}
\sphinxAtStartPar
Element Type: \sphinxcode{\sphinxupquote{Column}} / Group Name: \sphinxcode{\sphinxupquote{G.Column}}

\sphinxlineitem{탄성기둥}
\sphinxAtStartPar
Element Type: \sphinxcode{\sphinxupquote{Column}} / Group Name: \sphinxcode{\sphinxupquote{E.Column}}

\sphinxlineitem{벽체}
\sphinxAtStartPar
Element Type: \sphinxcode{\sphinxupquote{Shear Wall}} / Group Name: \sphinxcode{\sphinxupquote{Wall}}

\sphinxlineitem{벽체 회전각 게이지}
\sphinxAtStartPar
Element Type: \sphinxcode{\sphinxupquote{Deformation Gage}} / Group Name: \sphinxcode{\sphinxupquote{Wall Rotation}} / Gage Type: \sphinxcode{\sphinxupquote{Wall type, rotation or shear}}

\sphinxlineitem{벽체 축변형률 게이지}
\sphinxAtStartPar
Element Type: \sphinxcode{\sphinxupquote{Deformation Gage}} / Group Name: \sphinxcode{\sphinxupquote{Wall Axial Strain}} / Gage Type: \sphinxcode{\sphinxupquote{Bar type, axial strain}}

\sphinxlineitem{기둥 회전각 게이지(X)}
\sphinxAtStartPar
Element Type: \sphinxcode{\sphinxupquote{Deformation Gage}} / Group Name: \sphinxcode{\sphinxupquote{Column Rotation(X)}} / Gage Type: \sphinxcode{\sphinxupquote{Beam type, rotation}}

\sphinxlineitem{기둥 회전각 게이지(Y)}
\sphinxAtStartPar
Element Type: \sphinxcode{\sphinxupquote{Deformation Gage}} / Group Name: \sphinxcode{\sphinxupquote{Column Rotation(Y)}} / Gage Type: \sphinxcode{\sphinxupquote{Beam type, rotation}}

\end{description}\end{quote}

\item {} 
\sphinxAtStartPar
Import하지 않지만, 이 후의 모델링 과정에서 사용자가 직접 생성할 Elements와 Gages의 그룹도 같은 방법으로 생성할 수 있습니다.
\begin{quote}\begin{description}
\sphinxlineitem{Imbedded Beam}
\sphinxAtStartPar
Element Type: \sphinxcode{\sphinxupquote{Beam}} / Group Name: \sphinxcode{\sphinxupquote{Imbedded Beam}}

\sphinxlineitem{보 회전각 게이지}
\sphinxAtStartPar
Element Type: \sphinxcode{\sphinxupquote{Deformation Gage}} / Group Name: \sphinxcode{\sphinxupquote{Beam Rotation}} / Gage Type: \sphinxcode{\sphinxupquote{Beam type, rotation}}

\end{description}\end{quote}

\item {} 
\sphinxAtStartPar
Masses를 Import하기 위해 \sphinxguilabel{Import/Export Structure Data}을 클릭하고, 생성된 창에서 \sphinxkeyboard{\sphinxupquote{Import}}를 클릭합니다.

\item {} 
\sphinxAtStartPar
\sphinxguilabel{Masses} 탭을 클릭한 뒤, Choose mass pattern에서 방금 생성한(또는 사용자가 원하는) Mass Pattern을 선택합니다.
Specify name of text file에 앞서 생성한 \sphinxcode{\sphinxupquote{Mass.csv}} 파일의 경로를 입력합니다.

\newpage

\begin{center}
\noindent\sphinxincludegraphics[scale=0.9]{{3_import_masses_설정}.png}
\end{center}

\newpage

\item {} 
\sphinxAtStartPar
입력 후 \sphinxkeyboard{\sphinxupquote{Test}} \sphinxhyphen{} \sphinxkeyboard{\sphinxupquote{확인}}을 클릭하면 아래와 같이 Import할 Nodes가 표시됩니다.

\begin{center}
\noindent\sphinxincludegraphics[scale=0.8]{{3_import_masses_완료}.png}
\end{center}

\sphinxAtStartPar
\sphinxkeyboard{\sphinxupquote{OK}}를 클릭하여 Import를 완료합니다.

\end{enumerate}
\end{sphinxShadowBox}

\sphinxstepscope


\section{Frame 생성}
\label{\detokenize{3_create_frames:frame}}\label{\detokenize{3_create_frames::doc}}
\sphinxstepscope


\section{Constraints, Supports 생성}
\label{\detokenize{3_create_supports:constraints-supports}}\label{\detokenize{3_create_supports::doc}}
\sphinxstepscope


\section{Sections 생성}
\label{\detokenize{3_create_sections:sections}}\label{\detokenize{3_create_sections::doc}}
\sphinxstepscope


\section{Drifts 생성}
\label{\detokenize{3_create_drifts:drifts}}\label{\detokenize{3_create_drifts::doc}}
\sphinxstepscope


\section{Imbedded Beams 생성}
\label{\detokenize{3_create_imbedded_beams:imbedded-beams}}\label{\detokenize{3_create_imbedded_beams::doc}}
\sphinxstepscope


\section{Properties Import}
\label{\detokenize{3_import_properties:properties-import}}\label{\detokenize{3_import_properties::doc}}
\sphinxstepscope


\section{Properties 지정}
\label{\detokenize{3_assign_properties:properties}}\label{\detokenize{3_assign_properties::doc}}
\sphinxstepscope


\section{지진파 Import}
\label{\detokenize{3_import_seismic_waves:import}}\label{\detokenize{3_import_seismic_waves::doc}}
\sphinxstepscope


\chapter{비선형정적해석}
\label{\detokenize{4_nonlinear_static_analysis:id1}}\label{\detokenize{4_nonlinear_static_analysis::doc}}
\sphinxstepscope


\chapter{비선형동적해석}
\label{\detokenize{5_nonlinear_dynamic_analysis:id1}}\label{\detokenize{5_nonlinear_dynamic_analysis::doc}}
\sphinxstepscope


\chapter{해석결과 후처리}
\label{\detokenize{6_post-processing:id1}}\label{\detokenize{6_post-processing::doc}}

\section{비선형해석 결과 출력}
\label{\detokenize{6_post-processing:id2}}

\section{결과 후처리 및 확인}
\label{\detokenize{6_post-processing:id3}}

\section{재해석 및 보강 설계}
\label{\detokenize{6_post-processing:id4}}

\section{보고서 작성}
\label{\detokenize{6_post-processing:id5}}


\renewcommand{\indexname}{색인}
\printindex
\end{document}